%!TEX root = ../Testprotokolle.tex

\chapter{Testdurchführung (2015-07-07)}

Art des Tests: Integrationstest\\
Abgedeckte Funktionen: \textbf{F110}
Beteiligte Tester: Carl Schiller\\

\section{Testumgebung}

Die Funktion wurde unter OS X Yosemite 10.10.4 auf einem Webserver  getestet. Es wurde eine deutsche Systemumgebung verwendet. Die Applikation wurde über Safari 8.0.7 gestartet.

\section{Testprotokoll}

\begin{longtable}{|p{4cm}|p{11cm}|}
\hline
\textbf{Funktion} & \textit{\textbf{F110}} \\
\hline
\textbf{Tester} & \textit{Carl Schiller} \\
\hline
\textbf{Eingaben} & \textit{Man wählt den Button "Story Mode" im Hauptmen\"u aus.} \\
\hline
\textbf{Soll - Reaktion} & \textit{W\"ahlt der Spieler zum ersten ma\"l den Story-Mode, so wird ein Tutorial gestartet. 
Das Tutorial wird gestartet und leitet den Spieler durch alle Elemente des Minispiel. Dazu geh\"oren zum einen der Belt, die Scrollcollection und das Dungeon. 
Es wird ebenfalls eine Einf\"uhrung in den SQL-Trainer gegeben. Das Tutorial wird auditiv begleitet.} \\
\hline
\textbf{Ist -- Reaktion} & \textit{Die Ist-Reaktion entspricht vollst\"andig der Soll-Reaktion.} \\
\hline
\textbf{Ergebnis} & \textit{Der Testlauf ist erfolgreich abgeschlossen worden.} \\
\hline
\textbf{Unvorhergesehene Ereignisse} &
\textit{} \\
\hline
\textbf{Nacharbeiten } & \textit{} \\
\hline
\end{longtable}

\section{Zusammenfassung}
Der Testlauf hat ergeben, dass das Tutorial wie gew\"unscht funktioniert.
\end{itemize}
