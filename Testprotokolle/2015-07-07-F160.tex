%!TEX root = ../Testprotokolle.tex

\chapter{Testdurchführung (2015-07-07)}

Art des Tests: Integrationstest\\
Testfall: T900\\
Abgedeckte Funktionen: \textbf{F160}
Beteiligte Tester: Denis Nagel\\

\section{Testumgebung}

Die Funktion wurde unter Windows 8.1 auf einem Webserver  getestet. Es wurde eine deutsche Systemumgebung verwendet. Die Applikation wurde über Firefox 39.0 gestartet.

\section{Testprotokoll}

\begin{longtable}{|p{4cm}|p{11cm}|}
\hline
\textbf{Funktion} & \textit{\textbf{F160}} \\
\hline
\textbf{Tester} & \textit{Denis Nagel} \\
\hline
\textbf{Eingaben} & \textit{Man wählt den Button "Punkte" im Abschnitt "Rankings" an.} \\
\hline
\textbf{Soll - Reaktion} & \textit{Es wird eine absteigend sortierte Liste der 10 Spieler mit den meisten erspielten Punkten, sowie die Position des aktuellen Nutzers angezeigt.} \\
\hline
\textbf{Ist -- Reaktion} & \textit{Es wurde eine absteigend sortierte Liste der 10 Spieler mit den meisten erspielten Punkten, sowie die Position des aktuellen Nutzers angezeigt.} \\
\hline
\textbf{Ergebnis} & \textit{Der Testlauf ist erfolgreich abgeschlossen worden.} \\
\hline
\textbf{Unvorhergesehene Ereignisse} &
\textit{} \\
\hline
\textbf{Nacharbeiten } & \textit{} \\
\hline
\end{longtable}

\newpage
\begin{longtable}{|p{4cm}|p{11cm}|}
\hline
\textbf{Funktion} & \textit{\textbf{F160}} \\
\hline
\textbf{Tester} & \textit{Denis Nagel} \\
\hline
\textbf{Eingaben} & \textit{Man wählt den Button "Lofi-Coins" im Abschnitt "Rankings" an.} \\
\hline
\textbf{Soll - Reaktion} & \textit{Es wird eine absteigend sortierte Liste der 10 Spieler mit den meisten eingesammelten Lofi-Coins, sowie die Position des aktuellen Nutzers angezeigt.} \\
\hline
\textbf{Ist -- Reaktion} & \textit{Es wurde eine absteigend sortierte Liste der 10 Spieler mit den meisten eingesammelten Lofi-Coins, sowie die Position des aktuellen Nutzers angezeigt.} \\
\hline
\textbf{Ergebnis} & \textit{Der Testlauf ist erfolgreich abgeschlossen worden.} \\
\hline
\textbf{Unvorhergesehene Ereignisse} &
\textit{} \\
\hline
\textbf{Nacharbeiten } & \textit{} \\
\hline
\end{longtable}

\newpage
\begin{longtable}{|p{4cm}|p{11cm}|}
\hline
\textbf{Funktion} & \textit{\textbf{F160}} \\
\hline
\textbf{Tester} & \textit{Denis Nagel} \\
\hline
\textbf{Eingaben} & \textit{Man wählt den Button "Zeit" im Abschnitt "Rankings" an.} \\
\hline
\textbf{Soll - Reaktion} & \textit{Es wird eine absteigend sortierte Liste der 10 Spieler mit der längsten Spielzeit, sowie die Position des aktuellen Nutzers angezeigt.} \\
\hline
\textbf{Ist -- Reaktion} & \textit{Es wurde eine absteigend sortierte Liste der 10 Spieler mit der längsten Spielzeit, sowie die Position des aktuellen Nutzers angezeigt.} \\
\hline
\textbf{Ergebnis} & \textit{Der Testlauf ist erfolgreich abgeschlossen worden.} \\
\hline
\textbf{Unvorhergesehene Ereignisse} &
\textit{} \\
\hline
\textbf{Nacharbeiten } & \textit{} \\
\hline
\end{longtable}

\newpage
\begin{longtable}{|p{4cm}|p{11cm}|}
\hline
\textbf{Funktion} & \textit{\textbf{F160}} \\
\hline
\textbf{Tester} & \textit{Denis Nagel} \\
\hline
\textbf{Eingaben} & \textit{Man wählt den Button "Durchläufe" im Abschnitt "Rankings" an.} \\
\hline
\textbf{Soll - Reaktion} & \textit{Es wird eine aufsteigend sortierte Liste der 10 Spieler mit der geringsten Anzahl an Durchläufen, sowie die Position des aktuellen Nutzers angezeigt.} \\
\hline
\textbf{Ist -- Reaktion} & \textit{Es wurde eine aufsteigend sortierte Liste der 10 Spieler mit den wenigsten Durchläufen, sowie die Position des aktuellen Nutzers angezeigt.} \\
\hline
\textbf{Ergebnis} & \textit{Der Testlauf ist erfolgreich abgeschlossen worden.} \\
\hline
\textbf{Unvorhergesehene Ereignisse} &
\textit{} \\
\hline
\textbf{Nacharbeiten } & \textit{} \\
\hline
\end{longtable}

\newpage
\begin{longtable}{|p{4cm}|p{11cm}|}
\hline
\textbf{Funktion} & \textit{\textbf{F160}} \\
\hline
\textbf{Tester} & \textit{Denis Nagel} \\
\hline
\textbf{Eingaben} & \textit{Man wählt den Button "SQL-Statements" im Abschnitt "Rankings" an.} \\
\hline
\textbf{Soll - Reaktion} & \textit{Es wird eine absteigend sortierte Liste der 10 Spieler mit den meisten gelösten SQL-Statements, sowie die Position des aktuellen Nutzers angezeigt.} \\
\hline
\textbf{Ist -- Reaktion} & \textit{Es wurde eine absteigend sortierte Liste der 10 Spieler mit den meisten gelösten SQL-Statements, sowie die Position des aktuellen Nutzers angezeigt.} \\
\hline
\textbf{Ergebnis} & \textit{Der Testlauf ist erfolgreich abgeschlossen worden.} \\
\hline
\textbf{Unvorhergesehene Ereignisse} &
\textit{} \\
\hline
\textbf{Nacharbeiten } & \textit{} \\
\hline
\end{longtable}

\newpage
\begin{longtable}{|p{4cm}|p{11cm}|}
\hline
\textbf{Funktion} & \textit{\textbf{F160}} \\
\hline
\textbf{Tester} & \textit{Denis Nagel} \\
\hline
\textbf{Eingaben} & \textit{Man wählt den Button "Erfolgsquote" im Abschnitt "Rankings" an.} \\
\hline
\textbf{Soll - Reaktion} & \textit{Es wird eine absteigend sortierte Liste der 10 Spieler mit der höchsten Erfolgsquote bei der Bearbeitung der SQL-Statements, sowie die Position des aktuellen Nutzers angezeigt.} \\
\hline
\textbf{Ist -- Reaktion} & \textit{Es wurde eine absteigend sortierte Liste der 10 Spieler mit der höchsten Erfolgsquote bei der Bearbeitung der SQL-Statements, sowie die Position des aktuellen Nutzers angezeigt.} \\
\hline
\textbf{Ergebnis} & \textit{Der Testlauf ist erfolgreich abgeschlossen worden.} \\
\hline
\textbf{Unvorhergesehene Ereignisse} &
\textit{} \\
\hline
\textbf{Nacharbeiten } & \textit{} \\
\hline
\end{longtable}

\section{Zusammenfassung}

Der Testlauf hat gezeigt, dass alle Funktionen des Ranking-Screens wie gewünscht funktionieren.
