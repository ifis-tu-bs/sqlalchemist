%!TEX root = ../Testprotokolle.tex

\chapter{Testdurchführung (2015-07-10)}

Art des Tests: Integrationstest\\
Abgedeckte Funktionen: \textbf{F140}
Beteiligte Tester: Carl Schiller\\

\section{Testumgebung}

Die Funktion wurde unter OS X Yosemite 10.10.4 auf einem Webserver  getestet. Es wurde eine deutsche Systemumgebung verwendet. Die Applikation wurde über Safari 8.0.7 gestartet.

\section{Testprotokoll}

\begin{longtable}{|p{4cm}|p{11cm}|}
\hline
\textbf{Funktion} & \textit{\textbf{F120}} \\
\hline
\textbf{Tester} & \textit{Carl Schiller} \\
\hline
\textbf{Eingaben} & \textit{Man wählt den Button "Story Mode" im Hauptmen\"u aus und im Anschluss gelangt man durch die T\"ur in das Dungeon.} \\
\hline
\textbf{Soll - Reaktion} & \textit{ W\"ahlt man den Story-Mode, so wird dem Spieler das Laboratory gezeigt. Nach dem ausw\"ahlen der T\"ur 
wird das Minispiel gestartet und die Spielfigur l\"auft kontinuierlich durch die Map. Bet\"atigt man die Kn\"opfe ,,N'' oder ,,M'' sollen die 
Soundeinstellung ge\"andert werden k\"onnen. \"Uber das Klicken auf die Potions, bzw. durch das Dr\"ucken der Tasten 1-n k\"onnen die 
Tr\"anke eingesetzt werden. Die Spielfigur wird dementsprechend schneller, sie springt h\"oher, stellt ihre Leben wieder her oder bekommt 
einen h\"oheren Verteidigungswert. Gelangt die Spielfigur an das Ende einer Map, so wird die n\"achste geladen. Ber\"uhrt die Spielfigur 
einen Gegner-Entit\"at, so verliert sie ein Leben. Springt sie hingegen auf sie drauf, so wird die Spielfigur nach vertikal nach oben geworfen. 
Ist der Spieler bis zur f\"unften map gekommen, so wird einen spezielle, un\"uberwindbare Map geladen. Sp\"atestens an dieser Stelle stirbt 
der Spieler an einem un\"uberwindbaren Hindernis und gelangt in den Game-Over-Screen. In diesen kann bereits vorher gelangen, indem 
er w\"ahrend seines Laufes im Dungeon alle Leben verliert. Im Game-Over-Screen wird dem Spieler aufgezeigt wie viele Lofi-Coins und wie 
viele Punkte er gesammelt hat. Auch die Tiefe des Minispiel und die entsprechenden Scroll die eingesammelt hat werden ihm hier gezeigt.} \\
\hline
\textbf{Ist -- Reaktion} & \textit{Die Ist-Reaktion entspricht vollst\"andig der Soll-Reaktion.} \\
\hline
\textbf{Ergebnis} & \textit{Der Testlauf ist erfolgreich abgeschlossen worden.} \\
\hline
\textbf{Unvorhergesehene Ereignisse} &
\textit{} \\
\hline
\textbf{Nacharbeiten } & \textit{} \\
\hline
\end{longtable}

\section{Zusammenfassung}
Der Testlauf hat ergeben, dass des Spielen des Minispiels vollst\"andig funktioniert.
\end{itemize}
