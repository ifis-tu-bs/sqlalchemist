%!TEX root = ../Pflichtenheft.tex

% Kapitel 6
%-------------------------------------------------------------------------------

\chapter{Nichtfunktionale Anforderungen}

\section{Funktionalität}
\begin{tabular}{|l|c|c|c|c|}
	\hline
	\textbf{Produktqualität} & \textbf{sehr gut} & \textbf{gut} & \textbf{normal} & \textbf{nicht relevant} \\ 
	\hline
	Angemessenheit           &                   &      x       &                 &                         \\ 
	\hline
	Richtigkeit              &         x         &              &                 &                         \\
	\hline
	Interoperabilität        &                  &      x       &                 &                         \\ 
	\hline
	Ordnungsmäßgkeit         &         x         &      x        &                 &                         \\ 
	\hline
\end{tabular}\\

Besonders penibel wird mit Fehlern in der SQL-Auswertung umgegangen, denn diese können falsche Lerneffekte erzeugen.
Die Webapplikation soll die geforderten Produktfunktionen (Spielesoftware) umsetzen. Für die Motivationsförderung der Benutzer 
wird es ein „Minispiel“ geben, welches in verschiedenen Modi gespielt werden kann. Für die Analyse sowie die Lösung der Aufgaben 
wird die externe Schnittstelle zum Teamprojekt verwendet. 

\section{Sicherheit}

\begin{tabular}{|l|c|c|c|c|}
	\hline
	\textbf{Produktqualität} & \textbf{sehr gut} & \textbf{gut} & \textbf{normal} & \textbf{nicht relevant} \\ \hline
	Zuverlässigkeit          &         x          &              &                 &                        \\ 
	\hline
	Reife                    &                  &       x       &                 &                         \\ 
	\hline
	Fehlertoleranz           &         x          &              &                &                         \\ 
	\hline
	Wiederherstellbarkeit    &                 &       x       &                 &                         \\ 
	\hline
\end{tabular}\\

Das Programm wird sehr zuverlässig sein, damit es in den Vorlesungs- und Hausaufgabenbetrieb eingebunden werden kann.  
Programminterne Fehler sind in Spielen ein großer Grund für Spaß- und Motivationsverlust und werden deshalb bestmöglich vermieden.

Die Sicherheit des Programms beschränkt sich auf den Zugriff zur Webapplikation. Der Zugriff ist nur durch die y-Nummer, beziehungsweise der 
E-Mail Adresse, und einem Passwort m\"oglich. 


\section{Benutzbarkeit}

\begin{tabular}{|l|c|c|c|c|}
	\hline
	\textbf{Produktqualität} & \textbf{sehr gut} & \textbf{gut} & \textbf{normal} & \textbf{nicht relevant} \\ \hline
	Verständlichkeit         &                   &      x       &                 &                         \\ 
	\hline
	Erlernbarkeit            &          x         &              &                &                         \\ 
	\hline
	Bedienbarkeit            &                   &             &       x          &                         \\ 
	\hline
	Effizienz                &                   &       x       &                 &                         \\ 
	\hline
	Zeitverhalten            &                   &      x       &                 &                         \\ 
	\hline
	Verbrauchsverhalten      &                   &      x       &                 &                         \\ 
	\hline
\end{tabular}\\

Da dies im Fokus eine Lernsoftware ist, sollte auch das Erlernen der Benutzung schnell zu beherrschen sein. 
Dabei muss der Nutzer Vorkenntnisse in SQL haben, um die Aufgaben verstehen zu können.
%Die Bedienbarkeit wird etwas zur Nebensache, da sie stark unter in sich komplexen SQL-Statements 
%leidet. Aber selbst die komplexesten Statements sollten möglichst schnell kontrolliert und ihr Ergebnis
%zurückgegeben werden, um den Spielspaß nicht zu vermindern.


\section{Änderbarkeit}

\begin{tabular}{|l|c|c|c|c|}
	\hline
	\textbf{Produktqualität} & \textbf{sehr gut} & \textbf{gut} & \textbf{normal} & \textbf{nicht relevant} \\ \hline
	Analysierbarkeit         &                   &      x       &                 &                         \\ 
	\hline
	Modifizierbarkeit        &                   &      x       &                 &                         \\ 
	\hline
	Stabilität               &         x         &              &                 &                         \\ 
	\hline
	Prüfbarkeit              &                   &             &        x         &                         \\ 
	\hline
	Übertragbarkeit          &       x      &              &                 &                         \\ 
	\hline
	Anpassbarkeit            &                  &              &                 &          x              \\ 
	\hline
	Installierbarkeit        &                   &             &                 &            x             \\ 
	\hline
	Konformität              &                   &             &                 &            x             \\ 
	\hline
	Austauschbarkeit         &                   &              &                &            x             \\ 
	\hline
\end{tabular} \\

Das Programm sollte sehr stabil laufen. Abstürze würden den Hausaufgabenbetrieb extrem behindern, was sowohl 
für die Lehrenden als auch für den Nutzer extrem ärgerlich wäre.
Die \"Ubertragbarkeit ist dadurch gegeben, dass es sich um eine Webapplikation handelt und somit in g\"angige Umgebungen portiert werden kann.
Der Quellcode des Programms sollte gut analysierbar sein, um Änderungen ohne großen
Zeitaufwand implementieren zu können. Mögliche Änderungen am Quellcode sollen keinen Einfluss auf die Stabilität der 
Spielsoftware haben, sodass ein fehlerfreier Betrieb gewährleistet werden kann. 


\section{Qualitätsanforderungen}

\begin{itemize}
\item  \qualityReq{10}{Das Produkt soll Benutzerfreundlich sein.}
\item  \qualityReq{20}{Das Programm soll zuverlässig und stabil laufen, um einen reibungslosen Ablauf der Hausaufgaben zu garantieren. \ref{F20}}
\item  \qualityReq{30}{Der SQL-Frage-Antwort Ablauf soll nach dem aktuellen SQL-Standard exakt kontrolliert werden.} 
\item  \qualityReq{40}{Das Produkt soll plattformunabhängig sein}
\end{itemize}
