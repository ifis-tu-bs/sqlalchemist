%!TEX root = ../Pflichtenheft.tex

\chapter{Produktdaten}

Produktdaten entsprechen s\"amtlichen relevanten Daten, die laufzeitpersistent gespeichert werden müssen.
Dabei werden für jeden Nutzer einzeln die  folgenden Daten benötigt:

\begin{data}{10}{Userdaten}
	Persönliche Daten des Nutzers:
	\begin{itemize}
		\item Profil (Referenz)
		\item E-Mail
		\item E-Mail bestätigt
		\item E-Mail Bestätigungscode
		\item y-Nummer (nur bei Studenten der TU BS)
		\item Matrikelnummer (nur bei Studenten der TU BS)
		\item Passworthash
		\item Passwort vergessen Code
		\item Nutzerberechtigung
	\end{itemize}
\end{data}

\begin{data}{20}{Sitzungsdaten}
	Verbindungsdaten zum Client:
	\begin{itemize}
		\item User (Referenz)
		\item Cookie-Identifikationsnummer
		\item IP-Adresse
		\item Erstellungsdatum (Timestamp)
		\item Auslaufdatum (Timestamp)
	\end{itemize}
\end{data}

\begin{data}{30}{Avatar}
	Avatar-Zuordnung:
	\begin{itemize}
		\item Name
		\item File-URL
		\item Skin-URL
		\item Preis
	\end{itemize}
\end{data}

\begin{data}{40}{Profildaten}
	Daten des Nutzerprofils:
	\begin{itemize}
		\item Username
		\item Avatar (Referenz)
		\item Münzen
		\item Musikeinstellungen
		\item Toneinstellungen
		\item Characterattribute (Health, Defense, Speed, Jump, Beltslots)
		\item Neue Story
	\end{itemize}
\end{data}

\begin{data}{50}{Avatarkauf}
	Speichert den Kauf eines Avatars:
	\begin{itemize}
		\item Profil (Referenz)
		\item Avatar (Referenz)
		\item Kaufdatum (Timestamp)
	\end{itemize}
\end{data}

\begin{data}{60}{Aufgabenpakete}
	Zusammenstellung mehrerer Aufgaben:
	\begin{itemize}
		\item Urheber (Referenz zu User)
		\item Paketname
		\item Paketbeschreibung
		\item Erstelldatum (Timestamp)
		\item Aufgabentyp (Hausaufgabe, allgemeines Trivia-Paket, Story-Paket)
		\item Lösungsbedingung (Begrenzung durch Zeit, Anzahl an Versuchen)
		\item Auslaufdatum (Timestamp)
		\item Zuffälige oder festgelegte Reihenfolge
	\end{itemize}
\end{data}

\begin{data}{70}{Aufgabenpaket-Fortschritt}
	Speicherung des Nutzerfortschrittes im Aufgabenpaket:
	\begin{itemize}
		\item Profil (Referenz)
		\item Aufgabenpaket (Referenz)
		\item Fortschritt (Aufgabennummer)
		\item Bearbeitungsbeginn (Timestamp)
		\item Bearbeitungsende (Timestamp)		
	\end{itemize}
\end{data}

\begin{data}{80}{Trank}
	Speichert die Daten der einzelnen Tränke:
	\begin{itemize}
		\item Trank-Art (Healing, Defense, Speed, Jump)
		\item Trank-Name
	\end{itemize}
\end{data}

\begin{data}{90}{Tasche}
	Inventar des Benutzers:
	\begin{itemize}
		\item Profil (Referenz)
		\item Trank (Referenz)
		\item Trank-Stärke
		\item Position im Gürtel
	\end{itemize}
\end{data}

\begin{data}{100}{Aufgabe}
	Einzelne SQL-Aufgaben:
	\begin{itemize}
		\item Urheber (Referenz zu User)
		\item Erstellungsdatum (Timestamp)
		\item Schwierigkeit
		\item Trank (Referenz)
		\item Rating
		\item Bearbeitungsdatum (Timestamp)
		\item Verfügbarkeit
		\item Vom Nutzer erstellt
	\end{itemize}
\end{data}

\begin{data}{110}{Aufgabe in Aufgabenpaket}
	Beschreibt die Aufgaben im Aufgabenpaket
	\begin{itemize}
		\item Aufgabe (Referenz)
		\item Aufgabenpaket (Referenz) 
		\item Reihenfolge
	\end{itemize}
\end{data}

\begin{data}{120}{Aufgabenbearbeitung}
	Zeigt, ob ein Benutzer eine Aufgabe bereits gelöst hat
	\begin{itemize}
		\item Profil (Referenz)
		\item Aufgabe (Referenz)
		\item Lösungsdatum (Timestamp)
		\item Erfolgreich
		\item Bearbeitungsdauer
	\end{itemize}
\end{data}

\begin{data}{130}{Aufgabenbewertung}
	Speichert Bewertungen zu Aufgaben von anderen Benutzern:
	\begin{itemize}
		\item Profil (Referenz)
		\item Aufgabe (Referenz)
		\item Bewertung
		\item Datum (Timestamp)
	\end{itemize}
\end{data}

\begin{data}{140}{Aufgabenkommentar}
	Speichert Benutzerkommentare zu Aufgaben
	\begin{itemize}
		\item Profil (Referenz)
		\item Aufgabe (Referenz)
		\item Kommentar
		\item Datum (Timestamp)
	\end{itemize}
\end{data}

\begin{data}{150}{Texte}
	Vorgefertigte Texte, die der narrative Character benutzt:
	\begin{itemize}
		\item Typ (Abhängig von Zeit oder Richtigkeit des Statements)
		\item Text
	\end{itemize}
\end{data}

\begin{data}{160}{Story-Text}
	Textelemente der Story:
	\begin{itemize}
		\item Text
		\item Bedingung
	\end{itemize}
\end{data}

\begin{data}{170}{Text in Aufgabenpaket}
	Texte für die Aufgabenpakete:
	\begin{itemize}
		\item Challenge (Referenz)
		\item Story-Text (Referenz)
		\item Reihenfolge
	\end{itemize}
\end{data}


\begin{data}{180}{Schriftrolle}
	Speicherung der verschiedenen Schriftrollen Arten:
	\begin{itemize}
		\item Name
		\item Typ (Rezept, Buff)
		\item Stärkeindikator 
		\item Attribut (Health, Defense, Speed, Jump)
		\item Potion (Referenz falls Typ = Rezept)
		\item Benutzt (Falls Typ = Buff)
	\end{itemize}
\end{data}

\begin{data}{190}{Spieler besitzt Schriftrolle}
	Speichert, welche Schrifrollen bereits eingesammelt wurden:
	\begin{itemize}
		\item Schriftrolle (Referenz)
		\item Profil (Referenz)
		\item Besitzt
	\end{itemize}
\end{data}










%Die langfristig zu speichernden Daten sind aus Benutzersicht detaillierter zu
%beschreiben. Dabei bietet sich eine formale Beschreibung an, um eine größere Präzisierung zu erreichen.

%Es kann die Darstellung gemäß Beispiel verwendet werden (alternativ kann auch ein Klassendiagramm mit entsprechender Beschreibung erstellt werden)

%Für jeden User müssen folgende Daten gespeichert werden:\\

%\begin{data}{10}{Lagerdaten}
%	Daten der Lagerplätze (max. 5.000):\\
%	-  Modulnummer,\\
%	-  Regalseite,\\
%	-  Regalspalte,\\
%	-  Regalzeile,\\
%	-  Fachhöhe,\\
%	-  Platzsperre (0 = nicht gesperrt, 1 = gesperrt für Einlagerung, 2 = gesperrt
%	   für Auslagerung, 3 = gesperrt für alle Zugriffe),\\
%	-  Reifenstatus (0 = frei,1 = reserviert für Einlagerung, 2= belegt, 3 =
%	   reserviert für Auslagerung),\\
%	-  Reifenseriennummer.\\
%\end{data}
%
%\begin{data}{20}{Moduldaten}
%	Daten der Module (max. 20):\\
%	-  Modulnummer,\\
%	-  Sperrkennzeichen (0 = nicht gesperrt, 1 = gesperrt für Einlagerung, 2 =
%	   gesperrt für Auslagerung, 3 = gesperrt für alle Zugriffe),\\
%	-  maximale Kapazität,\\
%	-  freie Kapazität,\\
%	-  belegte Plätze (ergibt sich aus Status und Zahl der zugeordneten
%	   Lagerplätze, wird aus Geschwindigkeitsgründen allerdings redundant
%	   mitgeführt).
%\end{data}
