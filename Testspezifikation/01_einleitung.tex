%!TEX root = ../Testspezifikation.tex

\chapter{Einleitung}

Einer der wichtigsten Aspekte bei der Entwicklung von Software ist das ausführliche Testen des Programms. Dabei müssen für diese Software alle Softwarebestandteile, auf ihre Funktionsweise, Ausfallsicherheit und Bedienbarkeit getestet werden. Gleichzeitig dient das Testen der Qualitätssicherung der Applikation. Dies ist speziell bei dieser Software wichtig, da sie den Lehrbetrieb der Veranstaltung \glqq Relationale Datenbanksysteme 1 \grqq~unterstützen und einen Teil des Hausaufgabenablaufs übernehmen soll.

Aus diesem Grund wird auch der SQL-Alchemist umfangreichen Tests unterzogen um einen reibungslosen Betrieb, während der späteren Nutzung zu gewährleisten. Insbesondere da das Programm später vorlesungsunterstützend eingesetzt werden soll und Einfluss auf die erfolgreiche Bearbeitung der Hausaufgaben der Studenten hat, ist es umso wichtiger, dass keine unvorhergesehenen Fehler auftreten, welche durch ein frühes Testen hätten verhindert werden können.

In diesem Dokument wird ausführlich auf die einzelnen Bereiche der Software eingegangen, welche verschiedenen Tests unterzogen werden müssen. Dabei wird auch die genaue Vorgehensweise der Testläufe näher beschrieben. 



%Wie in der Vorlesung erwähnt, ist das Testen von Software unerlässlich und darf
%bei keinem Softwareentwicklungsprozess fehlen.

%Diese Testdokumentation ist angelehnt an den IEEE 829-Standard, der als
%der bekannteste Standard für Software-Testdokumentationen gilt. 
%Der Standard definiert eine Menge von Testdokumenten
%und beschreibt deren Inhalte. 

%In diesem Kapitel soll kurz allgemein beschrieben werden, welche Software getestet wird, 
%welchen Qualitätsanforderungen sie genügt und welche Standards unter umständen zu beachten sind.
% (Umfang ca. $\frac{1}{3}$ bis 1 Seite)

%\textbf{Hinweis zu den Templates:}\\
%Dieses Template enthält Hinweise und Beispiele, die selbstverständlich zu entfernen sind.
% Angaben in <...> sind mit dem entsprechendem Text zu füllen.

%\textbf{Kapitel die bereits in der Abnahmetestspezifikation bearbeitet wurden, müssen hier nicht erneut
% bearbeitet werden. Es sollten jedoch die Annotationen umgesetzt worden sein.
% Es sollen also Kapitel 4-5 neu erarbeitet und Kapitel 2.1 überarbeitet werden.}


%\textbf{Dieses Kapitel kann aus der Abnahmetestspezifikation übernommen werden, 
%sollte jedoch die Bearbeitung der Annotationen beinhalten.}