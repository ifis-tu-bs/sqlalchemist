%!TEX root = ../Testspezifikation.tex

\chapter{Unit-Tests}
In den Unittests wird die Funktion der einzelnen Komponenten in sich getestet. Da dies für das Front-End, da in Javascript entwickelt, recht umständlich und zeitaufwändig ist, wurde dieser Testabschnitt durchgeführt. 
Ziel ist es sicher zu stellen, dass die programmierten Klassen und Methoden ihren jeweiligen Zweck zuverlässig erfüllen.

Testziel\ldots

\section{Zu testende Komponenten}

\begin{center}
	\begin{longtable}{|m{0,5cm}|m{6,29cm}|m{2,9cm}|m{4,65cm}|}
		\hline
		\textbf{Nr} & \textbf{Komponenten} & \textbf{Testfälle} & \textbf{Kommentar}\\ 
		\hline
		1  & C20, Back-End  & \hyperref[T1100]{T1100}   &     \\ 
		\hline
	\end{longtable}
\end{center}


\section{Testverfahren}
Die Tests wurden mit JUnit durchgeführt.

\subsection{Testskripte}
Es wurden keine speziellen Testskripte verwendet.

\section{Testfälle}

\begin{testcase}{1100}{Klasse x}
\label{T1100}
\item[Ziel]
Ziel ist es, die ordnungsgemäße Funktionstüchtigkeit der verschiedenen Klassen und Methoden des Back-Ends sicher zu stellen.
\item[Objekte/Methoden/Funktionen]
Die jeweiligen Methoden oder Klassen sind im jeweiligen Abschnitt in den Testprotokollen zu finden.
\item[Pass/Fail Kriterien]
Der Gesamttest gild als bestanden, wenn jede Klasse und MEthode ihren gewünschten Zweck erfüllt.
\item[Vorbedingung]
\item[Einzelschritte]
Zu jeder Klasse und Methode werden JUnit-Tests programmiert und die wichtigsten wurden in den Testdokumenten Protokolliert.
\item[Beobachtungen / Log / Umgebung]
\item[Besonderheiten]
\item[Abhängigkeiten]
\end{testcase}
