%!TEX root = ../Abnahmetestspezifikation.tex

\chapter{Testplan}

Im folgenden wird der Testplan für SQL-Alchemist beschrieben. Dabei wird sowohl auf die zu testenden Komponenten als auch die bereits im Pflichtenheft beschriebenen Funktionen und Merkmale eingegangen. Außerdem wird noch das genaue Vorgehen bei der Durchführung der Tests aufgezeigt.

\section{Zu testende Komponenten}

\begin{itemize}
\item \textbf{SQL Modul: }Das SQL Modul ist der Kern der Software. Der Zweck des Programms, nämlich der Unterstützung beim Erlernen der Datenbankanfragesprache SQL, wird hierüber realisiert. Daher muss besonders die korrekte Anzeige der Aufgaben und die automatische Bewertung getestet werden. Weiterhin soll eine simple Bedienbarkeit gewährleistet werden.
\item \textbf{Minispiel: }Das Minispiel stellt die zweite zentrale Komponente des Projektes dar und muss daher auch ausgiebig getestet werden. In diesem Fall liegt das Hauptaugenmerk auf der Spielbarkeit der einzelnen Level und die Einhaltung des Schwierigkeitsgrades.
\item \textbf{GUI: }Die grafische Oberfläche bildet die Schnittstelle zwischen Benutzer und Applikation. Daher muss hier geprüft werden, ob die Menüs auf allen Endgeräten korrekt angezeigt werden und intuitiv bedienbar sind.
\item \textbf{Datenbank: }Die Datenbank ist für die Speicherung aller Daten notwendig, die für die Erstellung der SQL-Abfragen benötigt werden. Demnach muss die Konsistenz und die Sicherung der Daten geprüft werden um einen einfachen Zugriff sicherzustellen.
\item \textbf{Benutzerverwaltung: }Die Benutzerverwaltung sorgt für die Speicherung der Spieleraccounts. In diesem Fall muss insbesondere die korrekte Rechtevergabe getestet werden.
\item \textbf{Ranglisten/Spielerprofile: }Die Profile halten den Fortschritt der Spieler fest, während die Ranglisten es ermöglichen, dass sich die Nutzer untereinander messen können. Daher ist es wichtig, dass alle zu erfassenden Daten in gewünschter Weise abgespeichert und angezeigt werden.
\item \textbf{Administrationstool: }Das Administrationstool ist das Werkzeug, welches benötigt wird damit die Dozenten und ihre Mitarbeiter das Spiel vorlesungsunterstützend einsetzen können. Die gesamte Hausaufgabenverwaltung läuft über dieses Tool. Demnach muss an dieser Stelle getestet werden, ob die Erstellung neuer Eingaben intuitiv und schnell machbar ist.
\end{itemize}

\section{Zu testende Funktionen/Merkmale}

\begin{itemize}
\item \textbf{$\langle$F10$\rangle$ Nutzer registrieren: }Neue Benutzer lassen sich durch Angabe von E-Mail/y-Adresse, sowie Passwort registrieren. Dabei wird auf die Vollständigkeit der Angaben geachtet und Studenten werden automatisch anhand der Eingaben erkannt.
\item \textbf{$\langle$F20$\rangle$ Nutzer anmelden: }Die Nutzer-Authentifizierung erkennt, ob es sich um einen registrierten Nutzer handelt und gewährt diesem Zugang.
\item \textbf{$\langle$F30$\rangle$ Nutzer abmelden: }Der Nutzer wird ordnungsgemäß abgemeldet und kann, ohne erneute Anmeldung, nicht mehr auf das Spiel zugreifen.
\item \textbf{$\langle$F40$\rangle$ Profil einsehen: }Das Profil wird auf allen Endgeräten korrekt und gut lesbar dargestellt. Alle angezeigten Daten sind richtig und aktuell.
\item \textbf{$\langle$F50$\rangle$ Benutzername ändern: }Der Benutzername wird erfolgreich geändert und gespeichert. Dabei ist zu prüfen, ob der Benutzername bereits vergeben wurde.
\item \textbf{$\langle$F60$\rangle$ Passwort ändern: }Das Passwort wird geändert, der Passworthash in der Datenbank aktualisiert sich dabei. Anschließend muss sich der Nutzer mit dem neuen Passwort anmelden können, das alte Passwort ist nicht mehr gültig.
\item \textbf{$\langle$F70$\rangle$ Avatar ändern: }Die Spielfigur im Minispiel und das Icon das Nutzers werden geändert und korrekt dargestellt.
\item \textbf{$\langle$F80$\rangle$ Benutzer löschen: }Nachdem der Nutzer seinen Account gelöscht hat oder dieser von einem Admin gelöscht wurde, wird der Nutzer abgemeldet und der Account in der Datenbank gelöscht. Seine erstellten Fragen und bearbeiteten Hausaufgaben bleiben erhalten. Ohne erneute Registrierung hat der Nutzer keinen Zugriff mehr auf das Spiel.
\item \textbf{$\langle$F90$\rangle$ Audioeinstellungen bearbeiten: }Soundeffekte und Musik können unabhängig voneinander ein- und ausgeschaltet werden. Die Einstellung wird korrekt in der Datenbank gespeichert und bei der nächsten Anmeldung korrekt geladen.
\item \textbf{$\langle$F100$\rangle$ Spielstand zurücksetzen: }Der Spielstand wird korrekt zurückgesetzt. Das Tutorial bleibt deaktiviert.
\item \textbf{$\langle$F110$\rangle$ Tutorial spielen: }Das Tutorial ist standardmäßig bei einem neuen Nutzer aktiviert, bis dieser das Tutorial erfolgreich beendet hat. Dann wird das Tutorial automatisch deaktiviert,  bis es im "`Settings"´-Menü erneut aktiviert wird.
\item \textbf{$\langle$F120$\rangle$ Story spielen: }Der Ablauf der Handlung ist korrekt und das Zusammenspiel von Minispiel und SQL-Modul funktioniert wie geplant.
\item \textbf{$\langle$F130$\rangle$ SQL-Trainer spielen: }Die Aufgaben wiederholen sich im Trivia-Modus nicht zu häufig. Im Story-Modus werden zu jedem Zeitpunkt die aktuellen Aufgaben gestellt. Zudem darf es nicht zu Fehlern kommen, die die Bearbeitung einschränken. Die Rückmeldung auf die Antwort der Nutzer ist korrekt. 
\item \textbf{$\langle$F140$\rangle$ Minispiel spielen: }Durch den zufälligen Aufruf der Level kommt es nicht zu Wiederholungen. Die Figur reagiert sofort auf Eingaben und das Spiel ist auf allen Endgeräten spielbar.
\item \textbf{$\langle$F150$\rangle$ Hausaufgaben bearbeiten: }Es werden die geforderten Aufgabenpakete geladen und die Bewertung funktioniert wie erwünscht. Bei der Bearbeitung darf es nicht zu Fehlern kommen, die den Betrieb einschränken.
\item \textbf{$\langle$F160$\rangle$ Ranglisten einsehen: }Die Ranglisten stellen alle geforderten Daten aktuell und richtig dar.
\item \textbf{$\langle$F170$\rangle$ Spieler suchen: }Spieler sollen anhand ihres Benutzernamens gesucht und korrekt angezeigt werden.
\item \textbf{$\langle$F180$\rangle$ Hausaufgabenergebnisse einsehen: }Die in den Hausaufgaben erzielten Ergebnisse werden ordnungsgemäß abgespeichert und für jeden Nutzer sind nur die eigenen Ergebnisse einsehbar.
\item \textbf{$\langle$F190$\rangle$ Benutzer befördern: }Die Option kann nur von einem Admin genutzt werden und befördert nur den richtigen Benutzer.
\item \textbf{$\langle$F200$\rangle$ Benutzer Adminrechte geben: }Die Option kann nur von einem Admin genutzt werden und ernennt nur den richtigen Benutzer zum Admin.
\item \textbf{$\langle$F210$\rangle$ Trivia-Aufgabe erstellen: }Nur berechtigte Nutzer haben Zugriff auf diese Funktion und die Aufgaben werden korrekt und vollständig in die Datenbank eingepflegt.
\item \textbf{$\langle$F220$\rangle$ Benutzeraufgaben bewerten: }Die Bewertung kann nur von Admins oder beförderten Nutzern vorgenommen werden und wird korrekt mit den bereits vorhandenen Bewertungen verrechnet.
\item \textbf{$\langle$F230$\rangle$ Hausaufgaben erstellen: }Hausaufgaben können nur von Admins erstellt und nur von Studenten bearbeitet werden. Die Hausaufgaben sind nur innerhalb des angegebenen Bearbeitungszeitraums freigeschaltet.
\item \textbf{$\langle$Q10$\rangle$ Benutzerfreundlichkeit: }Alle Funktionen sind für den Nutzer verständlich und können über möglichst kurze Wege durch das Menü erreicht werden.
\item \textbf{$\langle$Q20$\rangle$ Zuverlässigkeit: }Das Spiel stürzt nicht ab und die Bearbeitung der Hausaufgaben ist technisch jederzeit möglich.
\item \textbf{$\langle$Q30$\rangle$ Korrektheit: }Die Auswertung der Benutzereingaben ist korrekt.
\item \textbf{$\langle$Q40$\rangle$ Plattformunabhängigkeit: }Das Spiel ist plattformunabhängig nutzbar.
\end{itemize}
 

\section{Nicht zu testende Funktionen}

Bei allen von Dritten verwendeten Frameworks und Programmen werden ausreichende Testläufe vorausgesetzt. Daher werden keine weiteren Tests für die folgende Software vorgenommen:
\begin{itemize}
\item MelonJS
\item play
\item jQuery
\item AngularJS
\item Bootstrap
\item automatische Generierung von Schemata und Statements (Teamprojekt)
\item automatische Auftragsgenerierung (Bachelorarbeit)
\end{itemize}

\section{Vorgehen}

\textbf{a) Abnahme- und Funktionstests:} \\
Für den Abnahmetest ist folgendes Vorgehen geplant: Damit der Kunde einen guten Gesamtüberblick über alle Funktionalitäten erhält, sollen alle Anwendungsfälle aus der Anforderungsspezifikation in einer Reihenfolge getestet werden, in der auch spätere Nutzer die Applikation nutzen könnten. 
Dazu wird sich der Kunde zuerst einen neuen Account mit Adminrechten erstellen, mit dem er zunächst, unabhängig vom eigentlichen Programm, die verschiedenen Funktionen des Administrationstools testet. Danach wird er sich für die eigentliche Applikation anmelden um die unterschiedlichen zur Verfügung stehenden Funktionen auszuprobieren und auch die verschiedenen Spielmodi anzuspielen. Zuletzt wird der Kunde dann noch die Abmeldung durchführen.
Während dieses Testlaufs kann sich der Kunde dann auch gleich ein Bild davon machen, inwieweit die im Pflichtenheft festgelegten Qualitätsstandards eingehalten wurden.

\section{Testumgebung}
 Back-End Funktionen werden mit der JUnit getestet. Diese ist in das benutzte play!-Framework integriert.
 Front-End Funktionen werden manuell durchgeführt.

