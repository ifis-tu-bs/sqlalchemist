%!TEX root = ../Systementwurf2.tex

%Diese Datei dient der Versionskontrolle. Sie ist vollständig zu bearbeiten.

%----Überschrift------------------------------------------------------------
{\relsize{2}\textbf{Versionsübersicht}}\\[2ex]

%----Start der Tabelle------------------------------------------------------
\begin{longtable}{|m{1.78cm}|m{1.59cm}|m{2.86cm}|m{1.9cm}|m{5.25cm}|}

  \hline                                              % Linie oberhalb

  %----Spaltenüberschriften------------------------------------------------
  \textbf{Version}  &    \textbf{Datum}  &    \textbf{Autor}  &
  \textbf{Status}   &    \textbf{Kommentar}       \\  %Spaltenüberschrift
  \hline                                              % Gitterlinie

  %----die nachfolgeden beiden Zeilen so oft wiederholen und die ... mit den
  %    entsprechenden Daten zu füllen wie erforderlich
%   0.1    &    13.05.2014    &    F. Heymann   &        &   
%  Kapitel Datenmodell hinzugefügt\\
%  \hline
%   0.2    &    14.05.2014    &    P. Dittrich  &        &   
%  Kapitel Produktfunktionsanalyse und Glossar hinzugefügt\\
%  \hline 
%   0.3    &    14.05.2014    &    C. Sontag   &        &   
%  Einleitung hinzugefügt\\
%  \hline 
%   0.4    &    22.05.2014    &    C. Sontag   &        &   
%  Einleitung bearbeitet\\
%  \hline 
%   0.5    &    22.05.2014    &    D. Klose   &        &   
%  Kriterienerfüllung hinzugefügt\\
%  \hline 
%	 0.5    &    23.05.2014    &    A. Reiss   &        &   
%  Kapitel 3.1 hinzugefügt\\
%  \hline 
%	0.5    &    23.05.2014    &    A. Reiss   &        &   
%  Kapitel 3.2 hinzugefügt\\
%  \hline 
%    0.6    &    23.05.2014    &    P. Dittrich   &        &   
%  Einleitung leicht überarbeitet\\
%  \hline 
%	 0.6    &    23.05.2014    &    A. Reiss   &        &   
%  3.1/3.2 überarbeitet 3.3 hinzugefügt\\
%  \hline 
%	 0.6    &    23.05.2014    &    A. Reiss   &        &   
%  Kapitel 3 überarbeitet\\
%  \hline
%  	 0.7    &    26.05.2014    &    C. Sontag   &        &   
%  Kapitel 3.2 überarbeitet\\
%  \hline
%  0.8    &    28.05.2014    &    N. Widdecke   &   vorgelegt &  Layout und Rechtschreibkorrektur  \\
%  \hline 
  1.0    &    28.05.2014    &    Auftragnehmer &  vorläufig &  erster Entwurf des Dokuments mit den Kapiteln 1--4, 8 und 9\\
  \hline 
  2.0    &    28.05.2014    &    Auftragnehmer &  vorläufig &  Korrekturen\\
  \hline
  3.0 	& 05.07.2014 & Auftragnehmer & vorläufig & Einfügen der Kapitel 5--7\\
  \hline
    4.0 	& 05.07.2014 & Auftragnehmer & vorgelegt & Korrekturen \\
  \hline

%----Ende der Tabelle------------------------------------------------------
\end{longtable}

%Status: "`in Bearbeitung"', "`vorgelegt"', "`vorläufig"', "`abgenommen"' oder "`abgelehnt"'\\
%Kommentar: hier eintragen, was und wo geändert bzw. ergänzt wurde (z.B. Kapitel )
%
%
%Hinweis zum Template:
%Dieses Template enth\"ult Hinweise, die alle kursiv geschrieben sind. Alles
%Kursivgeschriebene ist selbstverst\"andlich bei Abgabe zu entfernen sind.
%Angaben in <\ldots> sind mit dem entsprechendem Text zu füllen.  Überzählige
%Kapitel, d.h. Kapitel, die nicht bearbeitet werden m\"ussen, da sie nicht der
%Aufgabenstellung entsprechen, bitte entfernen.
%
%Aufgabe des Grobentwurfs (Systemspezifikation 1): Aufgabe dieses Dokumentes ist es, die Architektur des
%Systems zu beschreiben und die daraus resultierenden Pakete durch die
%Definition von Schnittstellen zu Komponenten auszubauen.
%
%Aufgabe des Feinentwurfs (Systemspezifikation 2): Der Feinentwurf dokumentiert die klassischen Entwurfsentscheidungen wie z.B. Verwendung bestimmter Bibliotheken oder Entwurfsmuster. Darueber hinaus bildet der Feinentwurf die Grundlage der Implementierung, d.h. anhand
%dieses Dokumentes muss jeder Softwareentwickler in der Lage sein, das Produkt
%zu entwickeln. Es ist also auf Vollständigkeit der Dokumentation zu
%achten.
