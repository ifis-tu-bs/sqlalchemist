%!TEX root = ../Testspezifikation.tex

\chapter{Einleitung}

\NewsGenie ist eine Software zur sprachgesteuerten und interaktiven Auswahl von
Nachrichten aus durch den Benutzer vorausgewählten Quellen.

Mit natürlicher Sprache als Eingabemedium setzt \NewsGenie auf eine neue Art der
Interaktion mit dem Benutzer. Dieses Interface muss deshalb besonders stabil,
einfach und intuitiv sein, dabei aber tolerant auf Fehlbedienung reagieren.
Umfangreiche Tests sind für die Sicherung und Kontrolle dieser Ziele
unerlässlich um Fehler frühzeitig zu erkennen, die Spezifikationen zu
verifizieren und die Gesamtqualität der Software zu verbessern.

Die Software wird deshalb in allen Entwicklungsstadien umfangreichen Tests
unterzogen, um die im Pflichtenheft als besonders wichtig eingestuften
Qualitätsmerkmale wie Angemessenheit und Richtigkeit, Verständlichkeit,
Bedienbarkeit und Erlernbarkeit, sowie Zeitverhalten und Analysierbarkeit zu
gewährleisten.

Neben der Sprachsteuerung muss auch das Webinterface getestet werden. Es dient
dem Benutzer als Eingabeschnittstelle für Nachrichtenquellen und andere
Einstellungen. Hier gilt es neben der einfachen Bedienbarkeit auch die
Datensicherheit als Qualitätsmerkmal.


