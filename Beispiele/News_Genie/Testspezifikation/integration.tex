%!TEX root = ../Testspezifikation.tex

\chapter{Integrationstest}

Die Integrationstests überprüfen die reibungslose Zusammenarbeit der einzelnen Komponenten des \textit{NewsGenies}.

\section{Zu testende Komponenten}

Um die \textit{Query Processor}-Komponente auf die Korrektheit ihrer Integration
zu testen, müssen folgende Unterkomponenten getestet werden:
\begin{itemize}
\item \textit{Handler}
\item \textit{Natural Language Processing}
\item \textit{Analyser}
\item \textit{Searcher}
\item \textit{Result Processing}
\end{itemize}

Insgesamt sind im Paket \textit{Server} folgende Komponenten im
Zusammenspiel zu testen:
\begin{itemize}
\item \textit{Linked Open Data}
\item \textit{Datenbank}
\item \textit{Query Processor}
\item \textit{Webinterface}
\item \textit{Crawler}
\item \textit{Linked Open Data}
\item \textit{Cache}
\item \textit{Language}
\end{itemize}

Letztendlich ist die Kommunikation des \textit{Clients} mit dem \textit{Server} sicherzustellen:
\begin{itemize}
\item \textit{Client}
\end{itemize}

\section{Testverfahren}

Bei den Integrationstests wurde nach dem Bottom-Up Verfahren getestet. Dabei wurden zuerst die Komponenten des \textit{Query Processors} getestet und dieser anschliessend als eigene Komponente betrachtet. Danach werden wird die Integration der Komponenten des \textit{Servers} getestet und dieser ebenfalls anschliessend als eigene Komponente betrachtet. Abschliessend bleibt das Zusammenspiel von dem  \textit{Server} mit dem \textit{Client} zu testen.
Die Komponenten werden mit Ausnahme der Unterkomponenten des \textit{Query Processors} jeweils paarweise getestet. Diese werden auf Grund der starken Verzahnung und der parallelen Integration im Entwicklungsprozess in einem Testfall behandelt.

\subsection{Testskripte}
Es wurden keine Testskripte verwendet.

\pagebreak

\section{Testfälle}

\begin{testcase}{500}{\textit{QueryHandler} + \textit{Natural Language
Processing} +
\textit{Analyser} + \textit{Searcher} + \textit{Result Processing}}

\item[Ziel]
Ziel ist die Sicherstellung der Zusammenarbeit der Komponenten des \textit{Query Processors}.

\item[Objekte/Methoden/Funktionen]\
\textbf{Objekte: }
\begin{itemize}
\item \textit{ClientQuery}
\item \textit{AnalysedQuery}
\item \textit{SearchRequest}
\item \textit{SearchAnswer}
\item \textit{ClientAnswer}
\end{itemize}

\textbf{Methoden: }
\begin{itemize}
\item \textit{analyse()}
\item \textit{search()}
\item \textit{makeClientAnswer()}
\end{itemize}

\item[Pass/Fail Kriterien]\
Pass: Die Text des \textit{ClientQuerys} durchläuft die Komponenten und gibt
eine ClientAnswer zurück.\\
Fail: Es gibt eine Java Fehlermeldung oder der Text wurde nicht analysiert.
\item[Vorbedingung]
\begin{itemize} 
\item Der \textit{Query Handler} benötigt eine \textit{ClientQuery}, auf welche
reagiert wird.
\item Es muss die \textit{Datenbank} mit den entsprechenden Daten vorhanden
sein.
\end{itemize}
\item[Einzelschritte]
\begin{itemize}
\item \textit{QueryHandler} starten. 
\item Eine \textit{ClientQuery} an den \textit{QueryHandler} senden.
\item Die \textit{ClientAnswer} auf Korrektheit prüfen.
\end{itemize} 
\item[Beobachtungen / Log / Umgebung]
Es muss beobachtet werden ob die \textit{ClientQuery} tatsächlich alle Komponenten korrekt durchläuft. Dabei wird aus der \textit{ClientQuery} im \textit{Analyser} im Zusammenspiel mit \textit{Natural Language Processing} eine 
\textit{AnalysedQuery}. Anschliessend erstellt der \textit{Searcher} eine \textit{SearchRequest} an die \textit{Datenbank}und erhält eine \textit{SearchAnswer}. Letztendlich wird im \textit{Result Processing} eine \textit{ClientAnswer} generiert.
Zur Überprüfung wird nach dem korrekten Durchlauf die \textit{ClientAnswer}
überprüft.
\item[Besonderheiten]
Die \textit{ClientAnswer} kann leer sein, wenn die Datenbank nicht das
passende Suchergebnis beinhaltete.
\item[Abhängigkeiten] -
\end{testcase}

\pagebreak[2]

\begin{testcase}{600}{\textit{Query Processor} + \textit{Datenbank}}

\item[Ziel]\
Ziel ist die Sicherstellung der Zusammenarbeit des
\textit{Searchers} in der \textit{Query Processor}-Komponente und der
\textit{Datenbank}.
\item[Objekte/Methoden/Funktionen]\
\textbf{Objekte: }
\begin{itemize}
\item \textit{ClientQuery}
\item \textit{ClientAnswer}
\end{itemize}

\textbf{Methoden: }
\begin{itemize}
\item \textit{search()}
\item \textit{onReceive()}
\end{itemize}
\item[Pass/Fail Kriterien]\
Pass: Die Text des \textit{ClientQuerys} durchläuft die Komponenten des
\textit{Query Handlers} und gibt anschließend eine passende
\textit{ClientAnswer} aus der Datenbank aus.\\
Fail: Es gibt eine Java Fehlermeldung oder die Datenbank reagiert nicht.

\item[Vorbedingung]
\begin{itemize} 
\item Der \textit{Query Handler} benötigt eine \textit{ClientQuery}, auf welche
reagiert wird.
\item Die \textit{Datenbank} muss die entsprechenden Daten enthalten.
\end{itemize}
\item[Einzelschritte]
\begin{itemize}
\item \textit{QueryHandler} starten. 
\item Eine \textit{ClientQuery} an den \textit{QueryHandler} senden.
\item Die \textit{ClientAnswer} auf Korrektheit prüfen.
\end{itemize} 
\item[Beobachtungen / Log / Umgebung]
Es muss beobachtet werden, ob die \textit{ClientAnswer} tatsächlich die Daten
aus der \textit{Datenbank} enthält.
\item[Besonderheiten]
\begin{itemize}
\item Die \textit{ClientAnswer} kann leer sein, wenn die Datenbank nicht das
passende Suchergebnis beinhaltete.
\end{itemize}
\item[Abhängigkeiten] -
\textit{Query Processor}

\end{testcase}

\pagebreak[2]

\begin{testcase}{700}{\textit{Datenbank} + \textit{Crawler}}

\item[Ziel]\
Ziel ist die Sicherstellung der Zusammenarbeit der \textit{Datenbank} und des \textit{Crawlers}.
\item[Objekte/Methoden/Funktionen]\
\textbf{Objekte: }
\begin{itemize}
\item \textit{Articles}
\end{itemize}

\textbf{Methoden: }
\begin{itemize}
\item \textit{startCrawler()}
\item \textit{crawl()}
\end{itemize}

\item[Pass/Fail Kriterien]\
Pass: Der \textit{Crawler} arbeitet korrekt.\\
Fail: Es gibt eine Java Fehlermeldung oder Feeds werden nicht hinzugefügt.

\item[Vorbedingung]
\begin{itemize} 
\item Die \textit{Datenbank} muss Feeds enthalten.
\end{itemize}
\item[Einzelschritte]
\begin{itemize}
\item Starten des \textit{Crawlers}.
\item \textit{Crawler} nach 24 Stunden stoppen.
\item Artikel überprüfen.
\end{itemize} 
\item[Beobachtungen / Log / Umgebung]
Es muss beobachtet werden, ob eine Anzahl von Artikeln hinzugefügt wurden.
\item[Besonderheiten] -
\item[Abhängigkeiten] -

\end{testcase}

\pagebreak[2]

\begin{testcase}{800}{\textit{Datenbank} + \textit{Linked Open Data}}

\item[Ziel]\
Ziel ist die Sicherstellung der Zusammenarbeit der \textit{Datenbank} und \textit{Linked Open Data}.
\item[Objekte/Methoden/Funktionen]\
\textbf{Objekte: }
\begin{itemize}
\item \textit{FactAnswer}
\item \textit{FactRequest}
\end{itemize}

\textbf{Methoden: }
\begin{itemize}
\item \textit{searchFor()}
\end{itemize}
\item[Pass/Fail Kriterien]\ 
Pass: \textit{Linked Open Data} liefert das korrekte Ergebnis.\\
Fail: Es gibt eine Java Fehlermeldung oder einen Timeout.
\item[Vorbedingung] -
\item[Einzelschritte]
\begin{itemize}
\item Senden einer \textit{FactRequest} an die \textit{Datenbank}.
\item \textit{FactAnswer} überprüfen.
\end{itemize} 
\item[Beobachtungen / Log / Umgebung] -
\item[Besonderheiten] -
\item[Abhängigkeiten] -

\end{testcase}

\pagebreak[2]

\begin{testcase}{900}{\textit{Managment Handler} + \textit{Client}}

\item[Ziel]\
Ziel ist die Sicherstellung der Zusammenarbeit des \textit{Managment Handlers}
und des \textit{Clients}.
\item[Objekte/Methoden/Funktionen]\
\textbf{Objekte: }
\begin{itemize}
\item \textit{ClientQuery}
\item \textit{ClientAnswer}
\end{itemize}

\textbf{Methoden: }
\begin{itemize}
\item \textit{apply()}
\item \textit{onReceive()}
\end{itemize}
\item[Pass/Fail Kriterien]\
Pass: Die \textit{ClientAnswer} wird ausgegeben.\\
Fail: Es gibt eine Java Fehlermeldung, oder eine der folgenden Fehlermeldungen:
\begin{itemize}
\item \textit{``Remote actor not available [...]''}
\item \textit{``I'm sorry, that did not work.''}
\item \textit{``Not ready yet''}
\end{itemize}
\item[Vorbedingung]
Die \textit{Server} Architektur muss korrekt arbeiten.
\item[Einzelschritte]
\begin{itemize}
\item Senden einer \textit{ClientQuery} an den \textit{Managment Handler}.
\item \textit{ClientAnswer} überprüfen.
\end{itemize} 
\item[Beobachtungen / Log / Umgebung] -
\item[Besonderheiten] -
\item[Abhängigkeiten] -

\end{testcase}