%!TEX root = ../Testprotokolle.tex

\chapter{Testdurchführung (2014-07-01)}

Art des Tests: UnitTest\\
Ausgeführte Testfälle: \ref{T1300}\\ 
Beteiligte Tester: Philipp Dittrich\\
Getestete Funktionalitäten: Management-Cache füllen und gecachte Objekte
abrufen.

\section{Testumgebung}

Die Testfälle wurden unter MacOS X mit Eclipse SDk (Version 4.3.2) ausgeführt.

\section{Testprotokoll}

\begin{longtable}{|p{4cm}|p{11cm}|}
\hline
\textbf{Testfall} & \ref{T1300} \\
\hline
\textbf{Tester} & \textit{Beispiel: Philipp Dittrich} \\
\hline
\textbf{Eingaben} & \textit{Keine} \\
\hline
\textbf{Soll - Reaktion} & \textit{Der JUnit-Test wird ohne Fehler
abgeschlossen }
\\
\hline
\textbf{Ist -- Reaktion} & \textit{Der JUnit-Test bricht aufgrund einer
IndexOutOfBound-Exception in der Methode getLastOfType() ab.}
\\
\hline
\textbf{Ergebnis} & \textit{Es besteht ein Problem mit der Indexierung der
Listen in denen die Cache-Inhalte abgelegt werden.}
\\
\hline
\textbf{Nacharbeiten } & \textit{Bei der Umstellung des Caches auf eine neue
Methode zur Garbage Collection fiel ein bisher in den Listen gespeichertes
Objekt mit dem Zeitpunkt des letzten Cache-Zugriffes weg. Somit hat sich der
Index des Gesamtarrays um 1 verschoben. Dies führte in den betroffenen Methoden
zum Fehler. Durch Anpassung der Methoden an die neue Arraystruktur konnte
dieser Fehler behoben werden.}
\\
\hline
\end{longtable}

\section{Zusammenfassung}

Der Test hat gezeigt, dass die Klasse \textit{ManagementCache} fehlerhaft war.
Der Fehler konnte durch Anpassung der Berechung des Arrayindex behoben werden.
