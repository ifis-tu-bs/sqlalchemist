%!TEX root = ../Testprotokolle.tex

\chapter{Testdurchführung (2014-06-23)}

Art des Tests: UnitTest\\
Ausgeführte Testfälle: \ref{T1300}\\ 
Beteiligte Tester: Philipp Dittrich\\
Getestete Funktionalitäten: Management-Cache füllen und gecachte Objekte
abrufen.

\section{Testumgebung}

Die Testfälle wurden unter MacOS X mit Eclipse SDk (Version 4.3.2) ausgeführt.

\section{Testprotokoll}

\begin{longtable}{|p{4cm}|p{11cm}|}
\hline
\textbf{Testfall} & \ref{T1300} \\
\hline
\textbf{Tester} & \textit{Beispiel: Philipp Dittrich} \\
\hline
\textbf{Eingaben} & \textit{Keine} \\
\hline
\textbf{Soll - Reaktion} & \textit{Der JUnit-Test wird ohne Fehler
abgeschlossen }
\\
\hline
\textbf{Ist -- Reaktion} & \textit{Der JUnit-Test wird ohne Fehler
abgeschlossen}
\\
\hline
\textbf{Ergebnis} & \textit{Der Cache arbeitet wie vorgesehen.} \\
\hline
\textbf{Nacharbeiten } & \textit{Keine} \\
\hline
\end{longtable}

\section{Zusammenfassung}

Der Test hat gezeigt, dass die Klasse \textit{ManagementTest} unsere
Anforderungen an einen Cache zur Speicherung der bisher ausgetauschten
Nachrichten erfüllt.
