%!TEX root = ../Pflichtenheft.tex

% Kapitel 6
%-------------------------------------------------------------------------------

\chapter{Nichtfunktionale Anforderungen}

\section{Funktionalität}
\begin{tabular}{|l|c|c|c|c|}
	\hline
	\textbf{Produktqualität} & \textbf{sehr gut} & \textbf{gut} & \textbf{normal} & \textbf{nicht relevant} \\ \hline
	Angemessenheit           &         x         &              &                 &                         \\ \hline
	Richtigkeit              &         x         &              &                 &                         \\ \hline
	Interoperabilität        &                   &              &                 &             x            \\ \hline
	Ordnungsmäßigkeit        &                   &              &                 &             x          \\ \hline
	Sicherheit		 		 &                   &              &         x       &                         \\ \hline
\end{tabular}

	Da es sich um ein Nachrichtenprogramm handelt, ist es besonders wichtig, dass die Nachrichten aktuell sind.
	Wenn der Anwender direkt nach speziellen Informationen fragt, sollte dieser auch nur Nachrichten bekommen, die für ihn von Relevanz sind.
	Die Genauigkeit wird noch erhöht, indem der Anwender die Möglichkeit bekommt, eigene Informationsquellen anzulegen.
	Die Informationen werden dabei direkt und ohne Zensur aus den Nachrichten entnommen, weshalb diese immer korrekt die tatsächlichen Nachrichten widerspiegeln.
	Auf die Ordnungsmäßigkeit des Inhalts kann dabei jedoch nicht geachtet werden.
	Wenn sensible Anwenderdaten anfallen, müssen diese durch entsprechende Sicherheitsmechanismen geschützt werden.
	
\section{Zuverlässigkeit}
\begin{tabular}{|l|c|c|c|c|}
	\hline
	\textbf{Produktqualität} & \textbf{sehr gut} & \textbf{gut} & \textbf{normal} & \textbf{nicht relevant} \\ \hline
	Reife                    &                   &       x      &                 &                         \\ \hline
	Fehlertoleranz           &                   &       x      &                 &                         \\ \hline
	Wiederherstellbarkeit    &                   &              &        x        &                        \\ \hline
	Robustheit               &                   &       x      &                 &           				\\ \hline
\end{tabular}

Das Programm sollte reif genug sein alle geforderten Funktionen umzusetzen.
Bei fehlerhafter oder unverständlicher Eingabe muss das System stabil bleiben und ggf. eine für den Anwender verständliche Fehlermeldung ausgeben, oder Rückfragen stellen.

\section{Benutzbarkeit}
\begin{tabular}{|l|c|c|c|c|}
	\hline
	\textbf{Produktqualität} & \textbf{sehr gut} & \textbf{gut} & \textbf{normal} & \textbf{nicht relevant} \\ \hline
	Verständlichkeit         &        x          &              &                 &                         \\ \hline
	Erlernbarkeit            &        x          &              &                 &                         \\ \hline
	Bedienbarkeit            &        x          &              &                 &                         \\ \hline
\end{tabular}

Die Bedienung des Sprachinterfaces muss so einfach und intuitiv sein, dass es mit natürlicher Sprache, ohne  spezielle Kommandos, zu bedienen ist.
Das Webinterface muss selbst für normale Anwender ohne vorherige EDV-Vorkenntnisse bedienbar sein.
Alle für den Anwender verfügbaren Funktionen müssen unkompliziert, anwendbar und selbsterklärend sein.

\section{Effizienz}
\begin{tabular}{|l|c|c|c|c|}
	\hline
	\textbf{Produktqualität} & \textbf{sehr gut} & \textbf{gut} & \textbf{normal} & \textbf{nicht relevant} \\ \hline
	Zeitverhalten            &          x        &              &                 &                         \\ \hline
	Verbrauchsverhalten      &                   &              &        x        &                         \\ \hline
\end{tabular}

Auf Benutzereingaben soll innerhalb von fünf Sekunden reagiert werden.
Alle beim Anwender ausgeführten Funktionen sollten so leichtgewichtig sein, dass sie auf kleinen Computern, wie z.B. dem Raspberry Pi laufen können.

\section{Änderbarkeit}
\begin{tabular}{|l|c|c|c|c|}
	\hline
	\textbf{Produktqualität} & \textbf{sehr gut} & \textbf{gut} & \textbf{normal} & \textbf{nicht relevant} \\ \hline
	Analysierbarkeit         &         x         &              &                 &                         \\ \hline
	Modifizierbarkeit        &                   &      x       &                 &                         \\ \hline
	Stabilität               &                   &              &         x       &                         \\ \hline
	Prüfbarkeit              &                   &              &         x       &                         \\ \hline
\end{tabular}

Da Software von Fremdanbietern eingebunden wird, muss die Software besonders analysierbar sein, 
um unter Umständen schnell Anpassungen vornehmen zu können.
Modifikationen, wie z.B. weitere Sprachpakete, oder zusätzliche Chat-Funktionen, sollten ohne großen Aufwand hinzufügbar sein.

\pagebreak[4]

\section{Konkrete Qualitätsanforderungen}

\qualityReq{10}{
Die Antwortzeit auf Benutzereingaben darf nicht mehr als fünf Sekunden betragen.
}
\qualityReq{20}{
Das Produkt muss anwenderfreundlich sein (intuitive Bedienbarkeit für Benutzer ohne EDV-Vorkenntnisse).
}
\qualityReq{30}{
Fehlerhafte Eingaben dürfen das Programm nicht zum Absturz bringen.
}
\qualityReq{40}{
Der Anwender soll nur für ihn relevante Nachrichten erhalten.
}
\qualityReq{50}{
Beim Anwender ausgeführte Programme müssen so ressourcenschonend sein, dass sie auch auf einem Raspberry Pi laufen können.
}
