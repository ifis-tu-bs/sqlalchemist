%!TEX root = ../Pflichtenheft.tex

% Kapitel 4
%-------------------------------------------------------------------------------

\chapter{Produktfunktionen}\label{test1}

\begin{function}{10}{Sprachausgabe}
\item[Geschäftsprozess:] Ausgabe von natürlicher Sprache
\item[Anforderung:] \lfk{1}
\item[Ziel:] Der Client gibt den Inhalt einer Sprachdatei akustisch aus.
\item[Vorbedingung:] Der Client erhält eine Sprachdatei, die ausgegeben werden soll.
\item[Nachbedingung Erfolg:] Die Sprachdatei wurde vom Client ausgegeben.
\item[Nachbedingung Fehlschlag:] Die Sprachdatei wurde nicht ausgegeben.
\item[Akteure:] Client
\item[Beschreibung:] Bei Eingang einer Sprachdatei wird diese akustisch in natürlicher Sprache ausgegeben.
\end{function} ~

\begin{function}{11}{Sprachaufnahme}
\item[Geschäftsprozess:] Aufnahme von natürlicher Sprache
\item[Anforderung:] \lfk{1}
\item[Ziel:] Benutzeranfragen in natürlicher Sprache werden aufgenommen und temporär gespeichert.
\item[Vorbedingung:] Der Benutzer betätigt den Aufnahmeknopf.
\item[Nachbedingung Erfolg:] Die Anfrage wurde vom Client aufgenommen und in einer Sprachdatei gespeichert.
\item[Nachbedingung Fehlschlag:] Die Anfrage konnte nicht aufgenommen werden.
\item[Akteure:] Client, Benutzer
\item[Beschreibung:] ~
\begin{enumerate}
\item Der Benutzer betätigt den Aufnahme-Knopf am Client.
\item Der Benutzer spricht seine Anfrage in das Mikrofon.
\item Die Anfrage wird in einer Sprachdatei zur Weiterverarbeitung gespeichert.
\end{enumerate}
\end{function} ~

\pagebreak[2]

\begin{function}{12}{Umwandlung von Sprache in Text}
\item[Geschäftsprozess:] Umwandlung einer Sprachdatei in einen String
\item[Anforderung:] \lfk{1}, \lfk{2}
\item[Ziel:] Der Inhalt einer Sprachdatei wird ausgelesen und in Form eines Strings zurückgegeben.
\item[Vorbedingung:] Der Client erhält eine Sprachdatei, die übersetzt werden soll.
\item[Nachbedingung Erfolg:] Ein String mit dem Inhalt der Sprachdatei wurde zurückgegeben.
\item[Nachbedingung Fehlschlag:] Die Sprachdatei wurde nicht in einen String übersetzt.
\item[Akteure:] Client, Google-Speech-API
\item[Beschreibung:] Bei Eingang einer Sprachdatei wird deren Inhalt in Text übersetzt und als String zurückgegeben. Hierfür wird die Google-Speech-API verwendet.
\end{function} ~

\begin{function}{13}{Umwandlung von Text in Sprache}
\item[Geschäftsprozess:] Umwandlung eines Strings in eine Sprachdatei
\item[Anforderung:] \lfk{1}, \lfk{2}
\item[Ziel:] Ein String mit natürlicher Sprache wird in eine Sprachdatei übersetzt.
\item[Vorbedingung:] Der Client erhält einen String, der übersetzt werden soll.
\item[Nachbedingung Erfolg:] Eine Sprachdatei mit dem Inhalt des Strings wurde zurückgegeben.
\item[Nachbedingung Fehlschlag:] Der String konnte nicht in eine Sprachdatei übersetzt werden.
\item[Akteure:] Client
\item[Beschreibung:] Bei Eingang eines Strings wird dieser in eine Sprachdatei übersetzt.
\end{function} ~

\begin{function}{20}{Anfrage stellen}
\item[Geschäftsprozess:] Stellen einer Anfrage an das System
\item[Anforderung:] \lfk{5}
\item[Ziel:] Der Benutzer stellt eine Anfrage an das System und erhält eine dementsprechende Antwort.
\item[Vorbedingung:] Der Benutzer ist am Client eingeloggt.
\item[Nachbedingung Erfolg:] Der Benutzer erhält eine, zu seiner Anfrage passende, Antwort.
\item[Nachbedingung Fehlschlag:] Der Benutzer wird gebeten, seine Anfrage zu präzisieren.
\item[Akteure:] Benutzer, Client, Query-Processor, Datenbank
\item[Beschreibung:] ~
\begin{enumerate}
\item Der Benutzer nennt dem Client seine Anfrage. \ref{F11}
\item Der Inhalt der Sprachdatei wird in einen String übersetzt. \ref{F12}
\item Der String wird von dem Query-Processor verarbeitet.
\item Die generierte Antwort wird in eine Sprachdatei übersetzt. \ref{F13}
\item Die Antwort wird dem Benutzer akustisch ausgegeben. \ref{F10}
\end{enumerate}
\end{function} ~

\begin{function}{30}{Client Login}
\item[Geschäftsprozess:] Login am Client
\item[Anforderung:] \lfk{1}, \lfk{2}
\item[Ziel:] Der Benutzer loggt sich am Client ein, um Queries stellen zu können.
\item[Vorbedingung:] Der Benutzer muss einen Account besitzen.
\item[Nachbedingung Erfolg:] Der Benutzer ist eingeloggt und kann Queries stellen.
\item[Nachbedingung Fehlschlag:] Der Benutzer ist nicht eingeloggt.
\item[Akteure:] Benutzer, Client, Datenbank
\item[Beschreibung:] ~
\begin{enumerate}
\item Der Benutzer nennt dem Client seinen Namen. \ref{F11}
\item Der Inhalt der Sprachdatei wird in einen String übersetzt. \ref{F12}
\item Der String wird mit der Benutzer-Tabelle der Datenbank abgeglichen.
\item Bei Übereinstimmung wird der Benutzer eingeloggt.
\end{enumerate}
\end{function} ~

\begin{function}{31}{Client Logout}
\item[Geschäftsprozess:] Logout aus dem Client
\item[Anforderung:] \lfk{1}, \lfk{2}
\item[Ziel:] Der Benutzer ist nicht länger am Client eingeloggt.
\item[Vorbedingung:] Der Benutzer muss am Client eingeloggt sein.
\item[Nachbedingung Erfolg:] Der Benutzer ist nicht mehr am Client eingeloggt.
\item[Nachbedingung Fehlschlag:] Der Benutzer ist weiterhin am Client eingeloggt.
\item[Akteure:] Benutzer, Client
\item[Beschreibung:] Sobald der Benutzer eine bestimmte Zeit inaktiv war, wird er automatisch aus dem Client ausgeloggt.
\end{function} ~

\begin{function}{40}{News-Crawling}
\item[Geschäftsprozess:] Abrufen neuer Nachrichten der Feeds
\item[Anforderung:] \lfk{5}
\item[Ziel:] Die Datenbank wird regelmäßig um neu erschienene Nachrichten erweitert.
\item[Vorbedingung:] Es gibt keine Vorbedingung.
\item[Nachbedingung Erfolg:] Für alle Feeds wurden alle neuen Nachrichten eingelesen und in der Datenbank gespeichert.
\item[Nachbedingung Fehlschlag:] Es wurden nicht alle neuen Nachrichten eingelesen.
\item[Akteure:] Backend, Datenbank
\item[Beschreibung:] In regelmäßigen Abständen wird für jeden in der Datenbank gespeicherten Feed die folgende Prozedur ausgeführt:
\begin{enumerate}
\item Der neueste Artikel des Feeds wird abgerufen.
\item Falls der Artikel in der Datenbank vorhanden ist, wird die Prozedur für den Feed beendet.
\item Ansonsten wird der neue Artikel in der Datenbank gespeichert und die Prozedur für den nächsten Artikel wiederholt.
\end{enumerate}
\end{function} ~

\begin{function}{50}{Registrierung}
\item[Geschäftsprozess:] Registrierung eines neuen Nutzers
\item[Anforderung:] \lfk{4}
\item[Ziel:] Ein neuer Nutzer erstellt einen Account, der in der Datenbank gespeichert wird.
\item[Vorbedingung:] Es gibt keine Vorbedingung.
\item[Nachbedingung Erfolg:] Der neue Account wurde in der Datenbank gespeichert.
\item[Nachbedingung Fehlschlag:] Es wurde kein neuer Account angelegt.
\item[Akteure:] Benutzer, Webinterface, Datenbank
\item[Beschreibung:] ~
\begin{enumerate}
\item Der Benutzer gibt seinen Namen, sein Passwort und seine E-Mail-Adresse an.
\item Sind entweder der Name oder die E-Mail-Adresse bereits in der Datenbank vorhanden, können diese nicht verwedet werden.
\item Der Name, die E-Mail-Adresse und das gehashte Passwort werden als neuer Account in der Datenbank gespeichert.
\end{enumerate}
\end{function} ~

\begin{function}{60}{Webinterface Login}
\item[Geschäftsprozess:] Login am Webinterface
\item[Anforderung:] \lfk{4}
\item[Ziel:] Der Benutzer loggt sich am Webinterface ein.
\item[Vorbedingung:] Der Benutzer muss einen Account besitzen.
\item[Nachbedingung Erfolg:] Der Benutzer ist eingeloggt und kann somit auf seine persönlichen Feeds zugreifen und diese verwalten.
\item[Nachbedingung Fehlschlag:] Der Benutzer ist nicht eingeloggt.
\item[Akteure:] Benutzer, Webinterface, Datenbank
\item[Beschreibung:] ~
\begin{enumerate}
\item Der Benutzer gibt seinen Namen und sein Passwort an.
\item Das Passwort wird mit einem passenden Algorithmus gehasht.
\item Die Daten werden mit der Datenbank abgeglichen.
\item Bei Übereinstimmung wird der Benutzer eingeloggt.
\end{enumerate}
\end{function} ~

\begin{function}{61}{Webinterface Logout}
\item[Geschäftsprozess:] Logout aus dem Webinterface
\item[Anforderung:] \lfk{4}
\item[Ziel:] Der Benutzer ist nicht länger am Webinterface eingeloggt.
\item[Vorbedingung:] Der Benutzer muss am Webinterface eingeloggt sein.
\item[Nachbedingung Erfolg:] Der Benutzer ist nicht mehr am Webinterface eingeloggt.
\item[Nachbedingung Fehlschlag:] Der Benutzer ist weiterhin am Webinterface eingeloggt.
\item[Akteure:] Benutzer, Webinterface
\item[Beschreibung:] Der eingeloggte Benutzer betätigt den \glqq Logout\grqq\ Knopf und ist nicht länger im Webinterface eingeloggt. Außerdem wird ein Benutzer nach einer bestimmten Zeit automatisch ausgeloggt.
\end{function} ~

\begin{function}{70}{Feed-Liste anzeigen}
\item[Geschäftsprozess:] Anzeige einer Liste der Feeds eines Benutzers
\item[Anforderung:] \lfk{4}
\item[Ziel:] Es wird eine Liste aller Feeds eines Benutzers angezeigt.
\item[Vorbedingung:] Der Benutzer muss am Webinterface eingeloggt sein.
\item[Nachbedingung Erfolg:] Eine Liste der Feeds des Benutzers wird angezeigt.
\item[Nachbedingung Fehlschlag:] Es wird keine Liste angezeigt.
\item[Akteure:] Webinterface, Datenbank
\item[Beschreibung:] Bei erfolgreichem Login am Webinterface wird dem Benutzer eine Liste seiner abonnierten Feeds angezeigt. Er hat die Möglichkeit weitere Feeds hinzuzufügen und abonnierte Feeds zu entfernen.
\end{function} ~

\begin{function}{71}{Feed hinzufügen}
\item[Geschäftsprozess:] Hinzufügen eines Feeds zur Abonnementliste
\item[Anforderung:] \lfk{4}
\item[Ziel:] Der Benutzer fügt der Liste seiner Feeds einen neuen Feed hinzu.
\item[Vorbedingung:] Der Benutzer muss am Webinterface eingeloggt sein.
\item[Nachbedingung Erfolg:] Ein neuer Feed wurde, falls noch nicht geschehen, in der Datenbank angelegt und der Benutzer mit ihm verknüpft.
\item[Nachbedingung Fehlschlag:] Es wurde kein neuer Feed angelegt.
\item[Akteure:] Benutzer, Webinterface, Datenbank
\item[Beschreibung:] ~
\begin{enumerate}
\item Der Benutzer gibt die URL des hinzuzufügenden Feeds in das dafür vorgesehene Textfeld ein.
\item Der Benutzer betätigt den Knopf \glqq Add\grqq.
\item Der Feed wird in der Datenbank hinzugefügt und mit dem Benutzer verknüpft.
\end{enumerate}
\end{function} ~

\begin{function}{72}{Feeds entfernen}
\item[Geschäftsprozess:] Entfernen eines Feeds aus der Abonnementliste
\item[Anforderung:] \lfk{4}
\item[Ziel:] Der Benutzer entfernt Feeds aus seiner Abonnementliste.
\item[Vorbedingung:] Der Benutzer muss am Webinterface eingeloggt sein.
\item[Nachbedingung Erfolg:] Die Verknüpfung zwischen den ausgewählten Feeds und dem Benutzer wurde aus der Datenbank gelöscht.
\item[Nachbedingung Fehlschlag:] Die Verknüpfungen wurden nicht gelöscht.
\item[Akteure:] Benutzer, Webinterface, Datenbank
\item[Beschreibung:] ~
\begin{enumerate}
\item Der Benutzer setzt in seiner Feed-Liste einen Haken an alle Feeds, die er entfernen möchte.
\item Der Benutzer betätigt den \glqq Unsubscribe Selected\grqq\ Knopf.
\item Der Benutzer wird aufgefordert, das Entfernen der Feeds zu bestätigen.
\item Bei Bestätigung werden die Verknüpfungen zwischen dem Benutzer und den Feeds aus der Datenbank gelöscht.
\end{enumerate}
\end{function} ~

\begin{function}{80}{Passwort ändern}
\item[Geschäftsprozess:] Ändern des Passworts
\item[Anforderung:] \lfk{4}
\item[Ziel:] Das Passwort eines Benutzers wird geändert.
\item[Vorbedingung:] Der Benutzer muss am Webinterface eingeloggt sein.
\item[Nachbedingung Erfolg:] Das neue Passwort des Benutzers wurde in der Datenbank gespeichert.
\item[Nachbedingung Fehlschlag:] Das Passwort des Benutzers wurde nicht verändert.
\item[Akteure:] Benutzer, Webinterface, Datenbank
\item[Beschreibung:] ~
\begin{enumerate}
\item Der Benutzer betätigt den \glqq Change Password\grqq\ Knopf.
\item Der Benutzer gibt sein bisheriges Passwort, sein neues Passwort und eine Bestätigung des neuen Passworts in die dafür vorgesehenen Textfelder ein.
\item Der Benutzer betätigt den \glqq Change Password\grqq\ Knopf.
\item Wenn der Hash des bisherigen Passworts mit dem Hash aus der Datenbank und die beiden neuen Passwörter übereinstimmen, wird der Hash des neuen Passworts in der Datenbank gespeichert.
\end{enumerate}
\end{function} ~

\begin{function}{81}{Passwort Wiederherstellung}
\item[Geschäftsprozess:] Wiederherstellung eines vergessenen Passworts
\item[Anforderung:] \lfk{4}
\item[Ziel:] Einem Benutzer, der sein Passwort vergessen hat, wird ein neues Passwort zugewiesen.
\item[Vorbedingung:] Der Benutzer muss einen Account besitzen.
\item[Nachbedingung Erfolg:] Der Benutzer hat ein neues, zufällig generiertes Passwort per E-Mail erhalten und der Hash des Passworts wurde in der Datenbank gespeichert.
\item[Nachbedingung Fehlschlag:] Das Passwort in der Datenbank wurde nicht geändert.
\item[Akteure:] Benutzer, Webinterface, Datenbank
\item[Beschreibung:] ~
\begin{enumerate}
\item Der Benutzer betätigt den \glqq Recover Password\grqq\ Knopf.
\item Der Benutzer gibt seinen Namen und seine E-Mail-Adresse in die dafür vorgesehenen Textfelder ein.
\item Der Benutzer betätigt den \glqq Request Password\grqq\ Knopf.
\item Der Benutzer erhält ein temporäres, zufällig generiertes Passwort per E-Mail zugeschickt.
\item Der Hash des temporären Passworts wird in der Datenbank gespeichert.
\item Beim nächsten Login am Webinterface wird der Benutzer aufgefordert, sein Passwort zu ändern.
\end{enumerate}
\end{function} ~

\begin{function}{90}{Rolle wechseln}
\item[Geschäftsprozess:] Annehmen der Rolle eines anderen Benutzers
\item[Anforderung:] \lfk{4}
\item[Ziel:] Ein Administrator nimmt die Rolle eines anderen Benutzers an.
\item[Vorbedingung:] Der Administrator muss am Webinterface eingeloggt sein.
\item[Nachbedingung Erfolg:] Der Administrator hat die Rolle des Benutzers angenommen.
\item[Nachbedingung Fehlschlag:] Der Administrator hat die Rolle nicht gewechselt.
\item[Akteure:] Administrator, Webinterface, Datenbank
\item[Beschreibung:] ~
\begin{enumerate}
\item Der Administrator gibt den Namen, des Benutzers dessen Rolle er einnehmen möchte, in das dafür vorgesehene Textfeld ein.
\item Der Administrator betätigt den \glqq Sudo\grqq\ Knopf.
\item Der Administrator nimmt die Rolle des angegebenen Benutzers an.
\end{enumerate}
\end{function} ~

\begin{function}{100}{Benutzer-Liste anzeigen}
\item[Geschäftsprozess:] Anzeige einer Liste aller Benutzer
\item[Anforderung:] \lfk{4}
\item[Ziel:] Einem Administrator wird eine Liste aller Benutzer angezeigt.
\item[Vorbedingung:] Der Administrator muss am Webinterface eingeloggt sein.
\item[Nachbedingung Erfolg:] Eine Liste aller Benutzer wird angezeigt.
\item[Nachbedingung Fehlschlag:] Es wird keine Liste angezeigt.
\item[Akteure:] Webinterface, Datenbank
\item[Beschreibung:] Einem Administrator wird eine Liste aller Benutzer angezeigt. Er hat die Möglichkeit in die Rolle eines Benutzers zu wechseln oder einen Benutzer zu löschen.
\end{function} ~

\begin{function}{110}{Benutzer löschen}
\item[Geschäftsprozess:] Löschen eines Benutzers
\item[Anforderung:] \lfk{4}
\item[Ziel:] Ein Benutzeraccount wird endgültig aus der Datenbank gelöscht.
\item[Vorbedingung:] Der Benutzer muss am Webinterface eingeloggt sein.
\item[Nachbedingung Erfolg:] Der Benutzeraccount wurde aus der Datenbank gelöscht und der Benutzer ist nicht länger eingeloggt.
\item[Nachbedingung Fehlschlag:] Der Account wurde nicht gelöscht und der Benutzer ist weiterhin eingeloggt.
\item[Akteure:] Benutzer, Webinterface, Datenbank
\item[Beschreibung:] ~
\begin{enumerate}
\item Der Benutzer betätigt den \glqq Delete Account\grqq\ Knopf.
\item Der Benutzer wird aufgefordert, das Löschen des Accounts zu bestätigen.
\item Bei Bestätigung wird der Account endgültig aus der Datenbank gelöscht.
\end{enumerate}
\end{function} ~

\begin{function}{120}{Text-Anfrage}
\item[Geschäftsprozess:] Stellen einer Anfrage am Webinterface
\item[Anforderung:] \lfk{4}
\item[Ziel:] Der Benutzer stellt eine Anfrage in Textform aus dem Webinterface heraus. Anschließend kann er einzelne Ergebnisse bestätigen oder verwerfen.
\item[Vorbedingung:] Der Benutzer muss am Webinterface eingeloggt sein.
\item[Nachbedingung Erfolg:] Dem Benutzer wird eine Liste von Artikeln angezeigt, die zu seiner Anfrage passen.
\item[Nachbedingung Fehlschlag:] Der Benutzer wird aufgefordert, seine Anfrge zu präzisieren.
\item[Akteure:] Benutzer, Webinterface, Datenbank
\item[Beschreibung:] ~
\begin{enumerate}
\item Der Benutzer gibt seine Anfrage in das dafür vorgesehene Textfeld ein.
\item Der Benutzer betätigt den \glqq Anfrage stellen\grqq~Knopf.
\item Dem Benutzer wird eine Liste der gefundenen Artikel angezeigt, in der er Artikel bewerten kann oder er wird aufgefordert, die Anfrage zu präzisieren.
\end{enumerate}
\end{function} ~
