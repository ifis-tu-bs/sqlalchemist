%!TEX root = ../Pflichtenheft.tex

\chapter{Produkteinsatz}

\section{Anwendungsbereiche}
The SQL Alchemist wird entwickelt, um den Umgang und das praktische Anwenden von SQL-Anfragen einfach und spielerisch üben, beziehungsweise lernen zu können.
Dies ist oftmals nicht ohne Weiteres möglich, da es ein großer Aufwand ist sich selbst einen Datenbankserver zu installieren, ihn mit Daten und eigenem geeigneten Schema 
zu füllen und sich dafür eigene Aufgaben auszudenken, sowie diese auf Korrektheit zu überprüfen.
Die Anwendung wird nicht explizit Bezug auf die Vorlesung nehmen, sondern weitestgehend autonom funktionieren, mit Ausnahme des „Homework-Mode“, der durch die 
Wissenschaftlichen Mitarbeiter der TU Braunschweig genutzt werden kann, um den Studenten der RDB1-Vorlesung SQL-Hausaufgaben zu stellen.
Dieser Modus ist nur durch die Anmeldung als Student mit bekannter y-Nummer erreichbar.

\section{Zielgruppen}
Die Studenten der TU Braunschweig sind die Hauptzielgruppe. Um das Modul „RDB1“ erfolgreich abzuschließen ist es notwendig, die geforderten Pflichthausaufgaben im 
Homework-Mode zu absolvieren. Damit die Studenten motivierter lernen können, enthält der Story-Modus ein „Jump\&Run“-Minigame. 
Will sich der Student jedoch optimal auf die Klausur vorbereiten, um die SQL-Anfragen zu vertiefen, ist es hilfreich so viel wie möglich zu üben. 
Dies ist durch den Trivia-Mode möglich, in dem der Spieler Aufgaben in verschiedenen, wählbaren Schwierigkeitsgraden bearbeiten kann.

Auch für die wissenschaftlichen Mitarbeiter ist die Anwendung von Nutzen. Die gestellten Hausaufgaben müssen nicht mehr von Hand einzeln kontrolliert werden, 
da die Hausaufgaben in der Anwendung selbst geprüft und automatisch als abgeschlossen markiert werden, wenn alle Aufgabenteile korrekt bearbeitet wurden.
Sie müssen so nur die wöchentlichen Aufgaben in die Datenbank eintragen.

Für die Professoren ist die Anwendung insofern interessant, dass sie einen neuen Weg darstellt, um die Studenten für etwas sehr theoretisches, aber vorlesungsrelevantes 
zu begeistern und ihnen so eine motivierende Möglichkeit geboten wird, diese wichtigen Fähigkeiten auf angenehme Weise zu verinnerlichen. Da SQL aber nicht nur an der 
Universität wichtig ist, sondern vor allem in der Arbeitswelt eine wesentliche Rolle spielt, spricht die Anwendung auch private Nutzer an. Diese können ihr Wissen auf diesem 
Gebiet erweitern und festigen oder sogar völlig neu zu erlernen.
Neue Nutzer, also keine Studenten, können sich mit ihrer E-Mail-Adresse anmelden. Für diese Gruppe fällt der Homework-Mode weg und ist somit nicht nutzbar. Eventuell 
könnte die Anwendung später auch von anderen Universitäten in ihrem Vorlesungsbetrieb unterstützend eingesetzt werden.
Das Minispiel ist nur in Verbindung mit dem Lernen verwendbar. Ein Einzelmodus nur für das Minispiel gibt es nicht.


\section{Betriebsbedingungen}
Um die Anwendung nutzen zu können, wird ein internetfähiges Endgerät mit installiertem Webbrowser benötigt, zum Beispiel ein Computer, Smartphone oder Tablet.
Durch die Plattformunabhängigkeit kann sie auch unterwegs auf dem Smartphone verwendet werden.
Da es zudem keine App ist, die auf das Gerät geladen werden muss, sind keine manuellen Aktualisierungen durch den Benutzer nötig. Alle Updates können durch die 
Administratoren durchgeführt werden, ohne dass sie erst durch eine Handlung des Nutzers wirksam gemacht werden müssen.
Der Homework-Mode erfordert während des Semesters wöchentliche Betreuung durch einen wissenschaftlichen Mitarbeiter, der die aktuellen Aufgaben in die Datenbank einfügt.
Auch die Serverapplikation, sprich das Admin-Tool, ist \"uber einen einfachen Internetbrowser erreichbar.


