%!TEX root = ../Pflichtenheft.tex

\chapter{Glossar}

Belt: Die vom Spieler ausgew\"ahlten Potions, die er im Dungeon verwenden kann.

Character-Sheet: Eine \"Ubersicht \"uber die Attribute der Spielfigur.

Dungeon: Eine Aneinanderreihung von Leveln, die das Minispiel repr\"asentieren. 

Inventory: Im Inventory wird die Anzahl der aktuell vorhandenen Potions gespeichert.

Scrollcollection: Beschreibt den Fortschritt des Spielers indem es die bereits gesammelten Scrolls speichert. 

Herausforderung im Dungeon: Jede f\"unfte Map besitzt eine solche Herausforderung. Sie ist nur \"uberwindbar wenn mehrere Attribute der Spielfigur auf erh\"ohten Stand sind. Beispiele w\"aren eine extrem weite Schlucht oder einen besonders gro{\ss}en Gegner.

Playerstatistics: Attribute des Spielers, die Auswirkungen auf das Minispiel haben (Sprungh\"ohe, Geschwindigkeit, Leben und St\"arke).

Potions: Die im Spiel verwendete Bezeichnung f\"ur Tr\"anke, die dem Spieler tempor\"are Vorteile
verschaffen. Sie werden durch das richtige Bearbeiten von SQL-Anfragen erstellt werden k\"onnen. 

Scrolls: Einmalig im Spiel im einzusammelnde Objekte. Besitzt man diese, ist man in der Lage durch das L\"osen von SQL-Statements Potion zu erstellen oder die Playerstatistics dauerhaft zu erh\"ohen. 

Scroll-/Coin Limit: Gesetztes Limit, das die Anzahl an einsammelbaren Coins beziehungsweise Scrolls pro Tag bestimmt.

Scenetexts: Texte, die die Story und das Tutorial erz\"ahlen.
