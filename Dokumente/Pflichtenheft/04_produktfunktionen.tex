%!TEX root = ../Pflichtenheft.tex

% Kapitel 4
%-------------------------------------------------------------------------------

\chapter{Produktfunktionen}\label{test1}

\section{Nutzer registrieren}
\begin{function}{10}{Nutzer registrieren}
\item[Anwendungsfall:] Ein neuer Benutzer möchte Zugang zum Spiel erhalten.
\item[Anforderung:] \ref{RM4}, \ref{RM8}, \ref{RC2}
\item[Ziel:] Der Benutzer wird registriert und erhält eigene Login-Daten.
\item[Vorbedingung:] Der Nutzer ist noch nicht registriert.
\item[Nachbedingung Erfolg:] Der Nutzer verfügt über eigene eindeutige Login-Daten und wird zum Login-Bereich weitergeleitet.
\item[Nachbedingung Fehlschlag:] Die eingegebenen Daten waren nicht vollständig und die Registrierung schlägt fehl; es wird eine entsprechende Fehlermeldung angezeigt.
\item[Akteure:] ~Nutzer und Programm
\item[Auslösendes Ereignis:] Der Nutzer hat die zum Spiel gehörende URL eingegeben und klickt auf „Sign Up“.
\item[Beschreibung:] ~
\begin{enumerate}
  \item  Der Nutzer gibt in die vorgesehenen Texteingabefelder einen Benutzernamen, ein Passwort, sowie eine E-Mail-Adresse oder y-Nummer ein.
  \item  Der Nutzer bestätigt seine Eingaben mit einem Klick auf einen Button.
  \item Verfügt der Nutzer über eine y-Nummer, so ist er automatisch verifiziert, andernfalls erhält er eine E-Mail mit einem Aktivierungslink, den er zur Verifikation anklicken muss.
  \item  Der Nutzer wird anschließend zum Login-Bereich weitergeleitet.
\end{enumerate}
\end{function}

\section{Nutzer anmelden}
\begin{function}{20}{Nutzer anmelden}
\item[Anwendungsfall:] Einloggen des Nutzers um Zugang zum Spiel zu erhalten.
\item[Anforderung:] \ref{RM8}, \ref{RC2}
\item[Ziel:] Der Nutzer wird angemeldet.
\item[Vorbedingung:] Der Nutzer muss sich registriert und dadurch einen Account erstellt haben.
\item[Nachbedingung Erfolg:]  Die eingegebenen Daten wurden als korrekt verifiziert und der Nutzer wird auf den Hauptbildschirm weitergeleitet.
\item[Nachbedingung Fehlschlag:] Die eingegebenen Daten waren nicht korrekt und dem Nutzer wird der Zugang verweigert; es wird eine entsprechende Fehlermeldung angezeigt.
\item[Akteure:] ~Nutzer und Programm
\item[Auslösendes Ereignis:] Der Nutzer hat die zum Spiel gehörende URL eingegeben und wurde zum Login-Bildschirm weitergeleitet.
\item[Beschreibung:] ~
\begin{enumerate}
  \item  Der Nutzer gibt seine Email-Adresse/y-Nummer in das dafür vorgegebene Feld ein.
  \item  Der Nutzer gibt sein Passwort in das dafür vorgesehene Feld ein.
  \item Der Nutzer klickt auf „Einloggen“.
  \item  Das System gleicht die eingegebenen Daten mit den Nutzerdaten in der Datenbank ab.
  \item Der Nutzer wird zum Hauptbildschirm weitergeleitet.
\end{enumerate}
\end{function}

\section{Nutzer abmelden}
\begin{function}{30}{Nutzer abmelden}
\item[Anwendungsfall:] Der Nutzer will das Spiel verlassen und sich abmelden.
\item[Anforderung:] -
\item[Ziel:] Der Nutzer wird abgemeldet.
\item[Vorbedingung:] Der Nutzer muss sich im Hauptmenü befinden.
\item[Nachbedingung Erfolg:]  Die Abmeldung ist erfolgt und der Nutzer wird zum Anmeldebereich weitergeleitet. Ohne erneute Anmeldung ist keine Aktivität im Spiel mehr möglich.
\item[Nachbedingung Fehlschlag:] Es gab einen Fehler beim Ausführen der Abmeldung und eine entsprechende Meldung wird ausgegeben.
\item[Akteure:] ~Nutzer und Programm
\item[Auslösendes Ereignis:] Der Nutzer hat im Hauptmenü auf den „Logout“-Button geklickt.
\item[Beschreibung:] ~
\begin{enumerate}
  \item  Der Nutzer wird abgemeldet und verlässt das Spiel
\end{enumerate}
\end{function}

\section{Profil einsehen}
\begin{function}{40}{Profil einsehen}
\item[Anwendungsfall:] Der Nutzer möchte seine bisher gemachten Fortschritte nachverfolgen und ruft zu diesem Zweck sein Profilfenster auf.
\item[Anforderung:] \ref{RM2}, \ref{RM4}, \ref{RS4}, \ref{RS5}
\item[Ziel:] Der Nutzer kann sein Profil einsehen.
\item[Vorbedingung:] Der Nutzer muss sich eingeloggt haben.
\item[Nachbedingung Erfolg:]  Dem Nutzer wird sein Profilfenster, in welchem dessen Fortschritte in Form von Statistiken festgehalten werden, angezeigt.
\item[Nachbedingung Fehlschlag:] Es gab einen Fehler beim Aufruf und eine entsprechende Meldung wird ausgegeben.
\item[Akteure:] ~Nutzer und Programm
\item[Auslösendes Ereignis:] Der Nutzer hat auf seinen Avatar geklickt.
\item[Beschreibung:] ~
\begin{enumerate}
  \item  Das Profilfenster wird geöffnet
  \item  Der Nutzer sieht seine Statistiken und Nutzerdaten ein.
\end{enumerate}
\end{function}

\section{Benutzernamen \"andern}
\begin{function}{50}{Benutzernamen ändern}
\item[Anwendungsfall:] Der Nutzer will seinen Benutzernamen ändern.
\item[Anforderung:] \ref{RM4}
\item[Ziel:] Der Benutzername, der im Spiel und in den Ranglisten verwendet wird, wird geändert.
\item[Vorbedingung:] Der Nutzer muss sich eingeloggt haben und sich im „Settings“-Menü befinden.
\item[Nachbedingung Erfolg:]  Dem Nutzer wird sein Profilfenster mit dem neuen Benutzernamen angezeigt.
\item[Nachbedingung Fehlschlag:] Es gab einen Fehler beim Ändern des Benutzernamens und eine entsprechende Meldung wird ausgegeben. Der Nutzer behält seinen alten Nutzernamen.
\item[Akteure:] ~Nutzer und Programm
\item[Auslösendes Ereignis:] Der Nutzer klickt auf den „Change Username“-Button.
\item[Beschreibung:] ~
\begin{enumerate}
  \item  Es öffnet sich ein Texteingabefenster, in dem der Nutzer seinen neuen Benutzernamen eingibt.
  \item  Er bestätigt seine Eingabe, anschließend prüft das Programm, ob der Benutzername verwendet werden darf (also noch nicht vergeben ist).
\end{enumerate}
\end{function}

\section{Passwort \"andern}
\begin{function}{60}{Passwort ändern}
\item[Anwendungsfall:] Ein Nutzer will sein Passwort ändern.
\item[Anforderung:] \ref{RM6}
\item[Ziel:] Das Passwort wird geändert.
\item[Vorbedingung:] Der Nutzer muss eingeloggt sein und sich im „Settings“-Menü befinden.
\item[Nachbedingung Erfolg:]  Der Nutzer hat sein neues Passwort zugewiesen bekommen und gelangt zurück ins „Settings“-Menü.
\item[Nachbedingung Fehlschlag:] Es gab einen Fehler und eine entsprechende Meldung wird ausgegeben. Der Nutzer behält sein bisheriges Passwort.
\item[Akteure:] ~Nutzer und Programm
\item[Auslösendes Ereignis:] Der Nutzer klickt im „Settings“-Menü auf den Button „Change Password“
\item[Beschreibung:] ~
\begin{enumerate}
  \item  Es öffnet sich ein Texteingabefeld, in das der Nutzer das neue Passwort eingibt.
  \item  Der Benutzer muss sein altes Passwort zur Verifikation in einem weiteren Texteingabefeld eingeben.
  \item Der Nutzer bestätigt seine Eingaben.
\end{enumerate}
\end{function}

\section{Avatar \"andern}
\begin{function}{70}{Avatar ändern}
\item[Anwendungsfall:] Ein Nutzer will seinen Avatar (Spielfigur) ändern.
\item[Anforderung:]\ref{RM4}, \ref{RS3}
\item[Ziel:] Der Avatar des Nutzers wird geändert.
\item[Vorbedingung:] Der Nutzer muss eingeloggt sein und sich im „Profile“-Menü befinden.
\item[Nachbedingung Erfolg:]  Der neue Avatar des Nutzers wird übernommen.
\item[Nachbedingung Fehlschlag:] Es gab einen Fehler und eine entsprechende Meldung wird ausgegeben. Der Nutzer behält seinen bisherigen Avatar.
\item[Akteure:] ~Nutzer und Programm
\item[Auslösendes Ereignis:] Der Nutzer klickt im „Profile“-Menü auf den Button „Change Avatar“.
\item[Beschreibung:] ~
\begin{enumerate}
  \item  Es wird eine Liste angezeigt, die alle möglichen Avatare enthält.
  \item  Der Nutzer wählt einen Avatar aus der Liste aus.
  \item Die Auswahl wird über einen Button bestätigt.
\end{enumerate}
\end{function}

\section{Benutzer l\"oschen}
\begin{function}{80}{Benutzer löschen}
\item[Anwendungsfall:] Ein Nutzer will seinen Account löschen.
\item[Anforderung:] \ref{RM4}, \ref{RM5}
\item[Ziel:] Der Nutzeraccount wird gelöscht.
\item[Vorbedingung:] Der Nutzer muss eingeloggt sein und sich im „Settings“-Menü befinden.
\item[Nachbedingung Erfolg:]  Der Account des Nutzers wurde, inklusive aller auf diesen bezogenen Daten (ausgenommen die von diesem Nutzer erstellten SQL-Abfragen), aus der Nutzerverwaltung gelöscht. Der Nutzer wird automatisch ausgeloggt.
\item[Nachbedingung Fehlschlag:] Es gab einen Fehler und eine entsprechende Meldung wird ausgegeben. Der Account bleibt unverändert.
\item[Akteure:] ~Nutzer und Programm
\item[Auslösendes Ereignis:] Der Nutzer klickt im „Settings“-Menü auf den Button „Delete User“.
\item[Beschreibung:] ~
\begin{enumerate}
  \item  Der Nutzer wird gefragt, ob er wirklich seinen Account löschen möchte.
  \item  Der Nutzer bestätigt sein Vorhaben und  sein Account wird gelöscht, außerdem wird er automatisch ausgeloggt.
\end{enumerate}
\end{function}

\section{Audioeinstellungen bearbeiten}
\begin{function}{90}{Audioeinstellungen bearbeiten}
\item[Anwendungsfall:] Ein Nutzer will die  Soundeffekte oder die Musik aktivieren/deaktivieren.
\item[Anforderung:] -
\item[Ziel:] Die Audioeinstellungen werden geändert.
\item[Vorbedingung:] Der Nutzer muss eingeloggt sein und sich im „Settings“-Menü befinden.
\item[Nachbedingung Erfolg:]  Die Audioeinstellungen wurden geändert und deren Status wurde im Back-End gespeichert.
\item[Nachbedingung Fehlschlag:] Es gab einen Fehler und eine entsprechende Meldung wird ausgegeben. Die bisherigen Einstellungen bleiben bestehen.
\item[Akteure:] ~Nutzer und Programm
\item[Auslösendes Ereignis:] Der Nutzer klickt im „Settings“-Menü auf den Button „Audio Settings“.
\item[Beschreibung:] ~
\begin{enumerate}
  \item  Es öffnet sich ein Fenster, in dem der Nutzer die Soundeffekte und die Musik unabhängig voneinander an- und abschalten kann. 
  \item  Der Nutzer bestätigt die Änderungen.
\end{enumerate}
\end{function}

\section{Spielstand zurücksetzen}
\begin{function}{100}{Spielstand zurücksetzen}
\item[Anwendungsfall:] Ein Nutzer möchte das Spiel erneut von vorne beginnen.
\item[Anforderung:] -
\item[Ziel:] Der Spielstand wird zurückgesetzt.
\item[Vorbedingung:] Der Nutzer muss eingeloggt sein und sich im „Settings“-Menü befinden.
\item[Nachbedingung Erfolg:]  Der Spielstand ist zurückgesetzt.
\item[Nachbedingung Fehlschlag:] Es gab einen Fehler und eine entsprechende Meldung wird ausgegeben. Der bisherige Speicherstand bleibt bestehen.
\item[Akteure:] ~Nutzer und Programm
\item[Auslösendes Ereignis:] Der Nutzer klickt im „Settings“-Menü auf den Button „Reset Story“.
\item[Beschreibung:] ~
\begin{enumerate}
  \item  Der Nutzer wird gefragt, ob er wirklich seinen Spielstand zurücksetzen möchte.
  \item  Der Nutzer bestätigt das Löschen.
\end{enumerate}
\end{function}

\section{Tutorial spielen}
\begin{function}{110}{Tutorial spielen}
\item[Anwendungsfall:] Ein Nutzer startet den "`Story-Mode"´.
\item[Anforderung:] -
\item[Ziel:] Der Nutzer lernt die Grundsteuerung des Spieles kennen.
\item[Vorbedingung:] Der Nutzer muss eingeloggt sein und den "`Story-Mode"´ gestartet haben. Außerdem muss das Tutorial aktiviert sein.
\item[Nachbedingung Erfolg:]  Der Nutzer hat das Tutorial erfolgreich absolviert und der Anfang der Story wird gestartet.
\item[Nachbedingung Fehlschlag:] Es gab einen Fehler und das Tutorial konnte nicht gestartet werden.
\item[Akteure:] ~Nutzer und Programm
\item[Auslösendes Ereignis:] Der Nutzer startet den „Story-Mode“.
\item[Beschreibung:] ~
\begin{enumerate}
  \item  Das Tutorial wird gestartet.
  \item  Der Nutzer erhält Anweisungen, die dieser befolgen muss.
  \item Nach Abschluss aller Aufgaben wird das Tutorial abgeschlossen.
\end{enumerate}
\end{function}

\section{Story spielen}
\begin{function}{120}{Story spielen}
\item[Anwendungsfall:] Ein Nutzer möchte den „Story-Mode“ spielen. 
\item[Anforderung:] \ref{RM7}
\item[Ziel:] Der Nutzer spielt die Story.
\item[Vorbedingung:] Der Nutzer muss eingeloggt sein und sich im Hauptmenü befinden.
\item[Nachbedingung Erfolg:]  Der Nutzer spielt den „Story-Mode“ bis er von selbst abbricht oder die Story bis zum Ende absolviert hat.
\item[Nachbedingung Fehlschlag:] Es gab einen Fehler und die Story konnte nicht gestartet werden. Eine entsprechende Meldung wird ausgegeben.
\item[Akteure:] ~Nutzer und Programm
\item[Auslösendes Ereignis:] Der Nutzer klickt im Hauptmenü auf den Button „Story-Mode“.
\item[Beschreibung:] ~
\begin{enumerate}
  \item  Der Story-Durchlauf wird gestartet.
  \item  Der Nutzer spielt die Story, bis er sie zu einem beliebigen Zeitpunkt abbricht.
\end{enumerate}
\end{function}

\section{SQL-Trainer spielen}
\begin{function}{130}{SQL-Trainer spielen}
\item[Anwendungsfall:] Der Nutzer spielt den „Trivia-“ oder „Story-Mode“ oder er bearbeitet seine Hausaufgaben.
\item[Anforderung:] \ref{RM3}
\item[Ziel:] Der Nutzer löst, abhängig vom Modus, zufällige oder vorher festgelegte Aufgaben.
\item[Vorbedingung:] Der Nutzer muss eingeloggt sein und einen der Spielmodi spielen.
\item[Nachbedingung Erfolg:]  Der Nutzer löst die verschiedenen Aufgaben und die Ergebnisse werden in einer Nachbesprechung angezeigt.
\item[Nachbedingung Fehlschlag:] Das Spiel weist einen Fehler beim Aufruf der Aufgaben auf und wirft eine Fehlermeldung aus.
\item[Akteure:] ~Nutzer und Programm
\item[Auslösendes Ereignis:] Der Nutzer startet einen der verschiedenen Spielmodi.
\item[Beschreibung:] ~
\begin{enumerate}
  \item  Das Nutzer wird zum SQL-Modul weitergeleitet.
  \item  Die erste Aufgabe wird angezeigt und vom Spieler bearbeitet.
  \item Es folgen weitere Aufgaben, bis das aktuelle Aufgabenpaket abgeschlossen ist.
  \item Dem Nutzer werden die Ergebnisse angezeigt.
\end{enumerate}
\end{function}

\section{Minispiel spielen}
\begin{function}{140}{Minispiel spielen}
\item[Anwendungsfall:] Der Nutzer spielt den "`Story-Mode"´.
\item[Anforderung:]\ref{RM1}, \ref{RS1}, \ref{RC5}
\item[Ziel:] Der Nutzer spielt das Minispiel und versucht die jeweiligen Level zu lösen.
\item[Vorbedingung:] Der Nutzer muss eingeloggt sein und den vorhergehenden SQL-Abschnitt gelöst haben. Zusätzlich muss er sich „Story-Mode“ befinden.
\item[Nachbedingung Erfolg:]  Der Nutzer spielt die jeweils gestellten Level und die Ergebnisse werden in einer Nachbedingung angezeigt. Dann wird er in den nächsten SQL-Abschnitt weitergeleitet.
\item[Nachbedingung Fehlschlag:] Das Spiel weist einen Fehler beim Aufruf des Minispiels auf und wirft eine entsprechende Meldung aus.
\item[Akteure:] ~Nutzer und Programm
\item[Auslösendes Ereignis:] Der Nutzer hat den vorhergehenden SQL-Abschnitt gelöst.
\item[Beschreibung:] ~
\begin{enumerate}
  \item  Das Minispiel wird gestartet.
  \item  Die ersten vier Level werden nacheinander vom Spieler gelöst.
  \item Ein Endlevel wird aufgerufen und vom Spieler gelöst.
\end{enumerate}
\end{function}

\section{Hausaufgaben bearbeiten}
\begin{function}{150}{Hausaufgaben bearbeiten}
\item[Anwendungsfall:] Der Nutzer ruft die ihm zugeordneten Hausaufgaben auf um diese zu bearbeiten.
\item[Anforderung:] \ref{RM3}, \ref{RM7}
\item[Ziel:] Der Nutzer bearbeitet die vom Spiel gestellten Aufgaben.
\item[Vorbedingung:] Der Nutzer muss eingeloggt sein und sich im „Homework-Mode“ befinden.
\item[Nachbedingung Erfolg:]  Der Nutzer bearbeitet die gestellten Aufgaben und die Ergebnisse werden in einer Nachbesprechung angezeigt.
\item[Nachbedingung Fehlschlag:] Es gab einen Fehler beim Aufruf der Aufgaben und eine entsprechende Meldung wird ausgegeben.
\item[Akteure:] ~Nutzer und Programm
\item[Auslösendes Ereignis:] Der Nutzer hat im „Homework“-Menü auf den Button „Do Homework“ geklickt.
\item[Beschreibung:] ~
\begin{enumerate}
  \item  Der Hausaufgabenmodus wird gestartet.
  \item  Die erste Aufgabe wird angezeigt und vom Spieler bearbeitet.
  \item Es folgen weiter Aufgaben solange bis das komplette Hausaufgabenpaket abgearbeitet wurde.
  \item Die Nachbesprechung wird angezeigt.
\end{enumerate}
\end{function}

\section{Ranglisten einsehen}
\begin{function}{160}{Ranglisten einsehen}
\item[Anwendungsfall:] Aufruf der Ranglisten durch den Nutzer.
\item[Anforderung:] \ref{RM4}, \ref{RM7}, \ref{RS4}, \ref{RS5}
\item[Ziel:] Die Ranglisten wird dem Nutzer angezeigt.
\item[Vorbedingung:] Der Nutzer muss sich eingeloggt haben und sich im Hauptmenü befinden.
\item[Nachbedingung Erfolg:]  : Dem Nutzer werden die Ranglisten angezeigt.
\item[Nachbedingung Fehlschlag:] Es gab einen Fehler beim Aufruf der Ranglisten und es wird eine entsprechende Meldung angezeigt.
\item[Akteure:] ~Nutzer und Programm
\item[Auslösendes Ereignis:] Der Nutzer hat auf den „Leaderboard“-Button geklickt.
\item[Beschreibung:] ~
\begin{enumerate}
  \item  Das Ranglistenfenster wird geöffnet.
\end{enumerate}
\end{function}

\section{Spieler suchen}
\begin{function}{170}{Spieler suchen}
\item[Anwendungsfall:] Der Nutzer möchte einen Nutzer anhand des Benutzernamens suchen.
\item[Anforderung:] \ref{RM4}
\item[Ziel:] Dem Nutzer werden die Ergebnisse des gesuchten Benutzers angezeigt.
\item[Vorbedingung:] Der Nutzer muss sich eingeloggt haben und sich im Hauptmenü befinden. Er muss den Benutzernamen kennen, nach dem er suchen will.
\item[Nachbedingung Erfolg:]  : Dem Nutzer wird den gefundenen Nutzer angezeigt.
\item[Nachbedingung Fehlschlag:] Es gab einen Fehler beim Suchen der Nutzer in den Ranglisten und es wird eine entsprechende Meldung angezeigt.
\item[Akteure:] ~Nutzer und Programm
\item[Auslösendes Ereignis:] Der Nutzer hat auf den „Search User“-Button geklickt.
\item[Beschreibung:] ~
\begin{enumerate}
  \item  Es öffnet sich ein Texteingabefeld, in dem der zu suchende Benutzername eingetragen wird.
  \item  Der Nutzer bestätigt seine Eingabe.
  \item  Ihm wird das Ergebnis der Suche angezeigt.
\end{enumerate}
\end{function}

\section{Hausaufgabenergebnisse einsehen}
\begin{function}{180}{Hausaufgabenergebnisse einsehen}
\item[Anwendungsfall:] Ein Nutzer will seinen Stand bei den Hausaufgaben abrufen.
\item[Anforderung:] \ref{RM4}, \ref{RM6}, \ref{RM7}
\item[Ziel:]  Die Hausaufgabenergebnisse werden angezeigt.
\item[Vorbedingung:] Der Nutzer muss sich im Administrationstool angemeldet haben.
\item[Nachbedingung Erfolg:]  Der Student erhält eine Übersicht über seine bisherigen Ergebnisse bei den Hausaufgaben.
\item[Nachbedingung Fehlschlag:] Bei der Ausgabe der Ergebnisse gab es einen Fehler und eine entsprechende Meldung wird ausgegeben.
\item[Akteure:] ~Nutzer und Programm
\item[Auslösendes Ereignis:] Der Nutzer klickt im Administrationstool auf „View Results“.
\item[Beschreibung:] ~
\begin{enumerate}
  \item  Es wird eine Liste angezeigt, die die Ergebnisse aller bisher bearbeiteten Hausaufgaben anzeigt.
\end{enumerate}
\end{function}

\section{Benutzer bef\"ordern}
\begin{function}{190}{Benutzer befördern}
\item[Anwendungsfall:] Ein Admin will einen Studenten oder einen regulären Nutzer befördern, damit dieser eigene SQL-Aufgaben erstellen darf.
\item[Anforderung:] \ref{RM4}, \ref{RM5}
\item[Ziel:] Der beförderte Nutzer kann eigenständig SQL-Aufgaben erstellen.
\item[Vorbedingung:] Der zu befördernde Nutzer muss registriert sein und der Admin muss Zugang zum Admin-Tool haben.
\item[Nachbedingung Erfolg:]  : Der Student oder regulärer Nutzer ist befördert und wird nun als beförderter Nutzer behandelt.
\item[Nachbedingung Fehlschlag:] Es gab einen Fehler und eine entsprechende Meldung wird ausgegeben.
\item[Akteure:] ~Admin und Programm
\item[Auslösendes Ereignis:] Der Admin hat im Admin-Tool auf den Button „Promote User“ geklickt.
\item[Beschreibung:] ~
\begin{enumerate}
  \item  Der Admin gibt in einem Texteingabefeld den Benutzernamen des zu befördernden Studenten oder regulären Nutzers ein.
  \item  Die Eingabe wird mit einem Button bestätigt.
\end{enumerate}
\end{function}

\section{Einem Benutzer Adminrechte geben}
\begin{function}{200}{Benutzer Adminrechte geben}
\item[Anwendungsfall:] Ein Admin will einem anderen Nutzer Adminrechte geben.
\item[Anforderung:] \ref{RM4}, \ref{RM5}
\item[Ziel:] Der ausgewählte Nutzer wird zum Admin.
\item[Vorbedingung:] Der zu befördernde Nutzer muss registriert sein und der Admin muss Zugang zum Admin-Tool haben.
\item[Nachbedingung Erfolg:]  Der Nutzer ist befördert und wird nun als Admin behandelt.
\item[Nachbedingung Fehlschlag:] Es gab einen Fehler und eine entsprechende Meldung wird ausgegeben.
\item[Akteure:] ~Admin und Programm
\item[Auslösendes Ereignis:] Der Admin hat im Admin-Tool auf den Button „Promote User“ geklickt.
\item[Beschreibung:] ~
\begin{enumerate}
  \item  Der Admin gibt in einem Texteingabefeld den Benutzernamen des zu befördernden Studenten oder regulären Nutzers ein.
  \item  Die Eingabe wird mit einem Button bestätigt.
\end{enumerate}
\end{function}

\section{Eine Trivia-Aufgabe erstellen}
\begin{function}{210}{Trivia-Aufgabe erstellen}
\item[Anwendungsfall:] Ein beförderter Nutzer will eigene Aufgaben für den "`Trivia-Mode"´ erstellen.
\item[Anforderung:] \ref{RM6}, \ref{RM7}
\item[Ziel:] Die bestehende Aufgabensammlung wird um die erstellte Aufgabe erweitert.
\item[Vorbedingung:] Der beförderte Nutzer muss sich im Admin-Tool angemeldet haben.
\item[Nachbedingung Erfolg:]  Der beförderte Nutzer hat seine Aufgabe erstellt und gelangt zurück ins Admin-Tool.
\item[Nachbedingung Fehlschlag:] Es gab einen Fehler und eine entsprechende Meldung wird ausgegeben.
\item[Akteure:] ~beförderter Nutzer und Programm
\item[Auslösendes Ereignis:] Der Nutzer klickt im Admin-Tool auf den Button „Create New Task“.
\item[Beschreibung:] ~
\begin{enumerate}
\item  Es öffnet sich ein Texteingabefeld, in das der Nutzer die Aufgabenstellung mit der entsprechenden Lösung eingibt.
\item  Anschließend bestätigt der Nutzer seine Eingaben.
\end{enumerate}
\end{function}

\section{Benutzeraufgaben bewerten}
\begin{function}{220}{Benutzeraufgaben bewerten}
\item[Anwendungsfall:] Ein Admin oder ein beförderter Nutzer möchte von anderen beförderten Benutzern erstellte Aufgaben bewerten.
\item[Anforderung:] \ref{RM4}, \ref{RM5}
\item[Ziel:] Von beförderten Nutzern erstellte Aufgaben sollen bewertet werden.
\item[Vorbedingung:] Der Admin oder beförderte Nutzer muss sich im Admin-Tool befinden.
\item[Nachbedingung Erfolg:]  Die Aufgabe ist bewertet und die neue Bewertung wird mit den bereits vorhandenen Bewertungen verrechnet und der Nutzer wird auf den Hauptbildschirm weitergeleitet.
\item[Nachbedingung Fehlschlag:] Es gab einen Fehler und eine entsprechende Meldung wird ausgegeben. 
\item[Akteure:] ~Admin (oder beförderter Nutzers) und Programm
\item[Auslösendes Ereignis:] Der Admin oder beförderte Nutzer klickt im Admin-Tool auf den „Rate Tasks“-Button.
\item[Beschreibung:] ~
\begin{enumerate}
  \item  Dem Admin oder befördertem Benutzer erscheint eine Liste an (von beförderten Nutzern erstellte) Fragen.
  \item  Beim Klick auf eine Frage öffnet sich ein Fenster, in dem die Bewertung vorgenommen werden kann.
\end{enumerate}
\end{function}

\section{Hausaufgaben erstellen}
\begin{function}{230}{Hausaufgaben erstellen}
\item[Anwendungsfall:] Ein Admin erstellt neue Hausaufgaben, die von den Nutzern bearbeitet werden sollen.
\item[Anforderung:] \ref{RM6}, \ref{RM7}
\item[Ziel:] Die Hausaufgaben müssen allen Studenten zugänglich sein.
\item[Vorbedingung:] Der Admin muss sich im Admin-Tool befinden.
\item[Nachbedingung Erfolg:]  Die Hausaufgabe ist erstellt, wird in der Datenbank gespeichert und kann von den Studenten abgerufen werden.
\item[Nachbedingung Fehlschlag:] Es gab einen Fehler und eine entsprechende Meldung wird ausgegeben. 
\item[Akteure:] ~Admin und Programm
\item[Auslösendes Ereignis:] Der Admin klickt im Admin-Tool auf den „Create Homework“-Button.
\item[Beschreibung:] ~
\begin{enumerate}
  \item  Der Admin gibt die Aufgabenstellung, die Lösung, die Anzahl möglicher Bearbeitungsversuche je Student und den Bearbeitungszeitraum in Eingabefelder ein.
  \item  Er bestätigt seine Eingabe.
\end{enumerate}
\end{function}