%!TEX root = ../Testspezifikation.tex

%----\"Uberschrift------------------------------------------------------------
{\relsize{2}\textbf{Versions\"ubersicht}}\\[2ex]

%----Start der Tabelle------------------------------------------------------
\begin{longtable}{|m{1.78cm}|m{1.59cm}|m{2.86cm}|m{1.9cm}|m{5.25cm}|}

  \hline                                              % Linie oberhalb

  %----Spalten\"uberschriften------------------------------------------------
  \textbf{Version}  &    \textbf{Datum}  &    \textbf{Autor(en)}  &
  \textbf{Status}   &    \textbf{Kommentar}  \\       %Spalten\"uberschrift
  \hline                                              % Gitterlinie

  %----die nachfolgeden beiden Zeilen so oft wiederholen und die ... mit den
  %i   entsprechenden Daten zu f\"ullen wie erforderlich
  0.1    &    28.04.2014    &    P. Dittrich    &        &   
  Einleitung und Testplan, sowie Komponentendiagramm eingefügt\\
  \hline                                              % Gitterlinie unten
  0.2    &    01.05.2014    &    F. Heymann    &        &   
  Kapitel 2.2, 2.3, 2.4, 2.5 hinzugefügt\\
  \hline                                              % Gitterlinie unten
  0.3    &    02.05.2014    &    N. Widdecke    &        &   
  Kapitel 3 hinzugefügt\\
  \hline
  0.4    &    02.05.2014    &    C. Sontag    &        &   
  Kapitel 3.3.1 hinzugefügt\\
  \hline  
  0.5    &    02.05.2014    &    P. Dittrich    &        &   
  Kapitel 3.2 hinzugefügt und 3.1 bearbeitet\\
  \hline
  0.6    &    03.05.2014    &    D. Klose    &        &   
  Kapitel 3.3.3 Test-Case <T201> bearbeitet\\
  \hline  
  0.7    &    03.05.2014    &    A. Reiss    &        &   
  Kapitel 3.3.2 Test-Case <T200> bearbeitet\\
  \hline  
  0.8    &    05.05.2014    &    Auftragnehmer    &        &   
  diverse Verbesserungen\\
  \hline
  1.0    &    05.05.2014    &    Auftragnehmer    &    vorläufig    &   
  Hinweise des Betreuers eingearbeitet\\
  \hline 
  1.1    &    16.07.2014    &    C. Sontag    &        &   
  Unit-Tests hinzugefügt\\
  \hline 
%----Ende der Tabelle------------------------------------------------------
\end{longtable}

%Status: "`in Bearbeitung"', "`vorgelegt"', "`vorläufig"', "`abgenommen"' oder "`abgelehnt"'\\
%Kommentar: hier eintragen, was und wo geändert bzw. ergänzt wurde (z.B. Kapitel )

