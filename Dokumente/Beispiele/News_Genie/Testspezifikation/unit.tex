%!TEX root = ../Testspezifikation.tex

\chapter{Unit-Tests}

Das Ziel der Unit-Tests ist es, jede Komponente auf ihre korrekte Funktionsweise
zu testen. Hierbei werden besonders die Teile des Systems, die
für den Nutzer bzw. für die Verarbeitung seiner Anfagen relevant sind,
ausgiebig getestet. In den Tests werden mögliche Anfragen und Methodenparameter
an die entsprechenden Methoden übergeben und die Ausgabe mit einem erwarteten
Ergebnis verglichen. 

Nicht getestet werden vorallem die Semantik und Funktionen,
welche aus fertigen Bibliotheken als fertige .jar in das Projekt importiert
werden.
Aber auch das Webinterface wird nicht gesondert getestet, da Typesafe Activator
nach jeder Veränderung die Website nochmals erstellt und überprüft wird, sowie der Searcher
und Result-Processor, da diese keine Verarbeitung sondern nur Verteilung und
Aufbereitung von Daten übernehmen.

\section{Zu testende Komponenten}

Im Paket \textit{Server} sind folgende Komponenten zu testen:
\begin{itemize}
\item \textit{Linked Open Data}
\item \textit{Datenbank}
\item \textit{Query Processor}
\item \textit{Crawler}
\item \textit{Language}
\item \textit{Cache}
\end{itemize}

Weiterhin ist das Verständnis für Eingaben und die Ausgabe des
\textit{Servers} sicherzustellen.

\section{Testverfahren}

Alle zu testenden Komponenten und Methoden werden mit JUnit-Tests getestet.
Diese werden automatisch bei jeden Build durchgeführt und brechen bei
Nichterfüllung den Build-Vorgang ab.

\subsection{Testskripte}

Folgende Tests sind für den Server auszuführen:
\begin{itemize}
  \item AnalyserTest.java
  \item DatabaseTest.java
  \item CrawlerTest.java
  \item LanguageTest.java
  \item LinkedOpenData.java
  \item ManagmentCacheTest.java
\end{itemize}

Für den Server sollen folgende Tests durchgeführt werden:
\begin{itemize}
  \item BasicAnalyzedRequestTest.java
\end{itemize}


\section{Testfälle}

\begin{testcase}{1000}{Klasse LinkedOpenData}
\item[Ziel] Das Ziel des Tests ist es, die Erstellung einer korrekten
Suchanfrage mit den Daten des Analyser zu überprüfen.
\item[Objekte/Methoden/Funktionen] Mit diesem Test wird die Methode searchfor
getestet. Dafür wird eine neue \textit{LinkedOpenData}-Instanz, wenn nicht schon
vorhanden angelegt.
\item[Pass/Fail Kriterien] Der Test war erfolgreich, wenn der JUnit-Test ohne
Failure durchgelaufen ist.
\item[Vorbedingung] Es besteht eine Internetverbindung
\item[Einzelschritte] Starten des JUnit-Testes und Überprüfung der Ausgabe
\item[Beobachtungen / Log / Umgebung] Es wird eine Entwicklungsumgebung für Java
benötigt.
\item[Besonderheiten] -
\item[Abhängigkeiten] -
\end{testcase}

\begin{testcase}{1100}{Klasse Database}
\item[Ziel] Das Ziel des Tests ist es, die Datenbank auf ihre Funktionalität zu
prüfen.
\item[Objekte/Methoden/Funktionen] Mit diesem Test werden die wesentlichen
Funktionen der Datenbank überprüft sowie die Antwortzeit.
Dafür wird eine neue \textit{Database}-Instanz mittels Akka-Actors erstellt.
\item[Pass/Fail Kriterien] Der Test war erfolgreich, wenn der JUnit-Test ohne
Failure durchgelaufen ist.
\item[Vorbedingung] Es besteht eine Verbindung zur Datenbank.
\item[Einzelschritte] Starten des JUnit-Testes und Überprüfung der Ausgabe 
\item[Beobachtungen / Log / Umgebung] Es wird eine Entwicklungsumgebung für Java
benötigt.
\item[Besonderheiten] Der Test erstellt Daten in der Datenbank, löscht diese
allerdings als einen weiteren Testpunkt wieder.
\item[Abhängigkeiten] Der Test ist abhängig von der Datenbank
\end{testcase}

\begin{testcase}{1200}{Klasse Crawler}
\item[Ziel] Das Ziel des Tests ist es, den Crawler auf seine Crawl-Eigenschaft
zu überprüfen.
\item[Objekte/Methoden/Funktionen] Mit diesem Test wird die wesentliche
Funktion des Cawlers überprüft.
Dafür wird eine neue \textit{Crawler}-Instanz, ein englischer und ein deutscher
Feed erstellt .
\item[Pass/Fail Kriterien] Der Test war erfolgreich, wenn der JUnit-Test ohne
Failure durchgelaufen ist.
\item[Vorbedingung] Es besteht eine Internetverbindung.
\item[Einzelschritte] Starten des JUnit-Testes und Überprüfung der Ausgabe 
\item[Beobachtungen / Log / Umgebung] Es wird eine Entwicklungsumgebung für Java
benötigt.
\item[Besonderheiten] Der Test überprüft nicht die Semantik der gecrawlten
Artikel, sondern nur, ob der Crawler Artikel empfängt.
\item[Abhängigkeiten] -
\end{testcase}

\begin{testcase}{1300}{Klasse Language}
\item[Ziel] Das Ziel des Tests ist es, die Sprachabfrage aus der Sprachdatei auf
ihre Funktionalität zu prüfen.
\item[Objekte/Methoden/Funktionen] Mit diesem Test werden die wesentlichen
Funktionen für die Sprachdateiauslese überprüft.
Dafür werden alle Sprachen durch die \textit{Language}-Klasse geladen.
\item[Pass/Fail Kriterien] Der Test war erfolgreich, wenn der JUnit-Test ohne
Failure durchgelaufen ist.
\item[Vorbedingung] -
\item[Einzelschritte] Starten des JUnit-Testes und Überprüfung der Ausgabe 
\item[Beobachtungen / Log / Umgebung] Es wird eine Entwicklungsumgebung für Java
benötigt.
\item[Besonderheiten] -
\item[Abhängigkeiten] Der Test ist abhängig von den Sprachdateien.
\end{testcase}

\begin{testcase}{1400}{Klasse Analyser und Natural Language Processing}
\item[Ziel] Das Ziel des Tests ist es, die Funktion der Erkennung der Eingaben
des Benutzers auf ihre Funktionalität zu prüfen.
\item[Objekte/Methoden/Funktionen] Mit diesem Test werden die Funktionen des
Analysers sowie des Natural Language Processing überprüft. Dafür werden
Instanzen von \textit{Standfort-NLP} und des \textit{Analysers} neu erstellt,
wenn diese nicht bereits vorhanden sind.
\item[Pass/Fail Kriterien] Der Test war erfolgreich, wenn der JUnit-Test ohne
Failure durchgelaufen ist.
\item[Vorbedingung] -
\item[Einzelschritte] Starten des JUnit-Testes und Überprüfung der Ausgabe 
\item[Beobachtungen / Log / Umgebung] Es wird eine Entwicklungsumgebung für Java
benötigt.
\item[Besonderheiten] -
\item[Abhängigkeiten] -
\end{testcase}

\begin{testcase}{1300}{Klasse ManagmentCache}
\item[Ziel] Das Ziel des Tests ist es, den Cache auf seine Funktion zu
überprüfen.
\item[Objekte/Methoden/Funktionen] Mit diesem Test werden die wesentlichen
Funktionen des Caches überprüft.
Dafür werden verschiedene Akka-Messages erstellt und in den Cache geladen, damit
die Funktionen des Caches benutzt werden können.
\item[Pass/Fail Kriterien] Der Test war erfolgreich, wenn der JUnit-Test ohne
Failure durchgelaufen ist.
\item[Vorbedingung] -
\item[Einzelschritte] Starten des JUnit-Testes und Überprüfung der Ausgabe 
\item[Beobachtungen / Log / Umgebung] Es wird eine Entwicklungsumgebung für Java
benötigt.
\item[Besonderheiten] -
\item[Abhängigkeiten] -
\end{testcase}

\begin{testcase}{1400}{Klasse Client}
\item[Ziel] Das Ziel des Tests ist es, die Übereinstimmung eines gegebenen
Srings mit einer Anzahl von Strings zu vergleichen
\item[Objekte/Methoden/Funktionen] Mit diesem Test wird die wesentliche
Funktion für den Vergleich von mehreren Strings überprüft.
Dafür wird eine Instanz der Klasse \textit{BasicAnalyzedRequest} erstellt und
genutzt.
\item[Pass/Fail Kriterien] Der Test war erfolgreich, wenn der JUnit-Test ohne
Failure durchgelaufen ist.
\item[Vorbedingung] -
\item[Einzelschritte] Starten des JUnit-Testes und Überprüfung der Ausgabe 
\item[Beobachtungen / Log / Umgebung] Es wird eine Entwicklungsumgebung für Java
benötigt.
\item[Besonderheiten] -
\item[Abhängigkeiten] -
\end{testcase}