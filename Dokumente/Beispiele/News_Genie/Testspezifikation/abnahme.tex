%!TEX root = ../Testspezifikation.tex

\chapter{Abnahmetest}

Der Abnahmetest für den \NewsGenie dient dazu, die vom im Pflichtenheft festgelegten Anforderungen an das Endprodukt auf ihre vollständige Erfüllung zu prüfen. 
Er soll sowohl die durch den Auftraggeber vorgestellten technischen Hintergrundprozesse sowie die eigentliche Bedienung des Produkts bewerten. 
Dies bedeutet, dass dem Kunden das vollständige Produkt mit all seinen Funktionen vorgestellt wird und er dieses mit seinen Vorstellungen abgleicht. 
Ziel ist es, dass der Kunde mit dem Endprodukt vollständig zufrieden ist und keine weiteren Anmerkungen hat, sodass das Produkt freigegeben werden kann.

\section{Zu testende Anforderungen}
\label{sec:anforderungen}

Der Kunde soll im Verlauf des Abnahmetests alle Funktionen, die der Nutzer später verwendet, ebenfalls ausführen. Dies bedeutet, dass er zunächst ein Benutzerkonto anlegt, um dann in beiden möglichen Arten Anfragen an den \NewsGenie zu stellen.
Die Reihenfolge soll dabei sein:

\begin{enumerate}

\item Registrierung \ref{F50}

\item Passwort wiederherstellen \ref{F81}

\item Webinterface Login \ref{F60}

\item Passwort ändern \ref{F80}

\item Feed-Liste anzeigen \ref{F70}

\item Feed hinzufügen \ref{F71}

\item Feed entfernen \ref{F72}

\item Administration: Benutzerliste anzeigen \ref{F100}

\item Administration: In Rolle eines Benutzers wechseln \ref{F90}

\item Administration: Benutzer löschen \ref{F110}

\item Text-Anfrage \ref{F120}

\item Webinterface Logout \ref{F61}

\item Client Login \ref{F30}, umfasst \ref{F11} und \ref{F12}

\item Anfrage stellen \ref{F20} (Sprachinterface)

\item Client Logout \ref{F31}

\end{enumerate}

Mit diesem Ablauf werden auch die Kriterien \ref{RM1}, \ref{RM2}, \ref{RM4}
\ref{Q10}, \ref{Q20}, \ref{Q30}, \ref{Q40} abgedeckt.

\section{Testverfahren}
\label{sec:testverfahren}

Der Abnahmetest wird in einem Plenum mit allen Stakeholdern erfolgen. Von den
Auftragnehmern wird zu Beginn der komplette Funktionsumfang von \NewsGenie
präsentiert. Dabei werden dem Kunden auch die Backend-Komponenten (Crawler,
QueryProcessor und Datenbank) im Detail vorgestellt und Wartungsaufgaben
besprochen.

Für den praktischen Teil des Abnahmetests bekommt der Kunde Gelegenheit weitere
Mitarbeiter oder externe Teilnehmer einzubeziehen. Diese sollen nun den
kompletten Lebenszyklus eines Benutzers aus Benutzer und Administrationssicht im
System nachzuvollziehen und zu testen. Dies erfolgt in zwei Teilen: An einem
Arbeitsplatz mit Bildschirm und Tastatur wird das Webinterface getestet,
anschließend bekommt der Kunde die Gelegenheit an einem bereitstehenden Client
das Sprachinterface zu testen.

Den Teilnehmern wird ein Testskript zur Verfügung gestellt, in dem die geplanten
Schritte kurz erläutert werden. Zunächst registriert der Kunde sich im
Webinterface von \NewsGenie \ref{F50}. Bevor er sich einloggt, soll die
Wiederherstellung eines vergessenen Passwortes von ihm getestet werden
\ref{F81}. Nach erfolgreicher Registrierung meldet sich der Kunde mit dem neu
erstellten Benutzer an der Weboberfläche an \ref{F60}. Nun soll er versuchen,
sein Passwort zu ändern \ref{F80}.

Im nächsten Schritt testet der Kunde die Funktionen zur Verwaltung seiner
persönlichen Nachrichtenquellen. Dazu lässt er sich seine RSS-Feed-Liste
anzeigen \ref{F70}, fügt anschließend einen RSS-Feed hinzu \ref{F71} und testet
abschließend das Löschen von RSS-Feeds \ref{F72}.

Zum Testen der Administrations-Funktionen von \NewsGenie wird dem Kunden nun ein
Benutzer mit entsprechenden Rechten zur Verfügung gestellt. Nun testet der Kunde
die Benutzerliste \ref{F100}, wechselt in die Rolle eines normalen Benutzers
\ref{F90} und kehrt in die Administrationsübersicht zurück, um einen Benutzer zu
löschen \ref{F110}.

Als letzter Test im Webinterface soll der Kunde nun eine Text-Anfrage an
\NewsGenie stellen \ref{F120} und die Anzeige der Ergebnisse im Webinterface in
Augenschein nehmen. Anschließend loggt er sich aus \ref{F61}.

Nun beginnt der Test des Clients. Der Kunde drückt zuerst den Sprach-Button um
sich per Sprachbefehl am Client einzuloggen \ref{F30}. Anschließend stellt er
verschiedene Anfragen an \NewsGenie \ref{F20}. Hierfür wird ein Skript der
Möglichkeiten mit Testvorschlägen erarbeitet, so dass der Kunde Gelegenheit hat,
alle Features kennenzulernen und zu testen. Abschließend meldet der Kunde sich,
wiederum per Sprachbefehl, am Client ab \ref{F31}.

\subsection{Testskripte}

Zur Durchführung des Abnahmetests werden dem Kunden ein erläuterter Ablaufplan wie in \ref{sec:anforderungen} und Vorschäge zu Sprachanfragen übergeben.
Weitere Testskripte sind nicht nötig.

\section{Testfälle}

\begin{testcase}{100}{Ein- und Ausgabe in natürlichen interaktiven
"`Gesprächen"'}

\item[Ziel]~\\
Zweck des Tests ist es, das Verständnis des \NewsGenies bei sprachlichen
Eingaben sowie seiner sprachlichen Ausgaben zu überprüfen. Auch wird hierbei
das interaktive Verhalten getestet. Gleichzeitig wird es ermöglicht, das Login
und Logout-System zu testen.

\item[Objekte/Methoden/Funktionen]~\\
\ref{RM1}, \ref{RM2}, \ref{F30}, \ref{F31}, \ref{F10}, \ref{F11},
\ref{F12}, \ref{F13}, \ref{F20}, \ref{Q10}, \ref{Q20}, \ref{Q30}, \ref{Q40}

\item[Pass/Fail Kriterien]~\\
Bedienung des Systems am Client durch eine nicht involvierte Testperson.
Test erfolgreich, wenn sich die Testperson erfolgreich anmelden kann, die
Testperson die Ausgabe, Benutzerfreundlichkeit
sowie Reaktionszeit als gut
bewertet und sich anschließend wieder abmelden kann. Die Benutzerfreundlichkeit
wird dabei an der Anzahl der nötigen Interaktionen bis zur Antwort gemessen
werden. Dagegen wird die Qualität der Antwort an der Qualität der Erkennung und
an dem Übereinstimmungsgrad der Antwort mit der Frage bestimmt. Auch muss die
Anfrage in einem fünf Sekunden Zeitraum geschehen.

\item[Vorbedingung]~\\
Die Datenbank muss mindestens 100 Einträg besitzen.

\item[Einzelschritte]~\\
Eingabe:
\begin{enumerate}
\item Anmeldung der Testperson am Client
\item \label{itm:second} Die Testperson stellt \NewsGenie eine verständliche
Frage oder unverständliche Frage
\item Die Testperson wartet bis \NewsGenie eine Antwort liefert
\item Die Antwort wird von der Testperson auf Korrektheit überprüft.
\item Soll noch eine Frage gestellt werden, so wendet die Testperson nochmals
alle Schritte ab Schritt \ref{itm:second} an.
\item Abmelden am Client
\end{enumerate}
Ausgabe:
Als Ausgabe erhalten wir eine positive oder negative Rückmeldung der Testperson
sowie einen Bogen mit den gestellten Fragen und den von \NewsGenie genannten
Antworten.

\item[Beobachtungen / Log / Umgebung]~\\
Wir beobachten weiterhin die Geschwindigkeit der Antwortsuche durch die Messung
der Zeit zwischen Fragestellung und Antwort und notieren
die gestellten Fragen und \NewsGenies Antworten.

\item[Besonderheiten]~\\
Es muss bei der Ausführung beachtet werden, dass jede Testperson eine andere
Stimme hat und dadurch Abweichungen entstehen können.

\item[Abhängigkeiten]~\\
Es gibt keine besonderen Abhängigkeiten.

\end{testcase}

\begin{testcase}{200}{Webinterface -- User}
\item[Ziel]~\\
Ziel ist es, das für den User bereitgestellte Webinterface auf seine Funktionalität zu prüfen.
Neben der Registration, dem Login und dem Logout, soll es dem User möglich sein Feeds hinzuzufügen und zu löschen. Darüber hinaus werden die Funktionen Passwort ändern, Passwort wiederherstellen, sowie das Löschen des Nutzeraccounts getestet.
Auch Textanfragen, welche über das Webinterface an den \NewsGenie gestellt werden können, werden auf ihre Funktionalität und Qualität getestet.

\item[Objekte/Methoden/Funktionen]~\\
\ref{RM4}, \ref{F50}, \ref{F60}, \ref{F61}, \ref{F70}, \ref{F71}, \ref{F72}, 
\ref{F80}, \ref{F81}, \ref{F110}, \ref{F120}, \ref{Q20} \\
Da die Qualitätskriterien \ref{Q10}, \ref{Q30}, \ref{Q40} bereits im ersten Testfall abgedeckt sind und die 
Technik der Text-/ und der Sprachanfrage die selbe ist, werden sie in diesem Testfall nicht berücksichtigt.
\item[Pass/Fail Kriterien]~\\
Eine nicht involvierte Testperson Registriert sich am Webinterface und Loggt sich ein. Anschliessend testet sie die verfügbaren Funktionen und bewertet die Funktionalität und Benutzerfreundlichkeit.
\item[Vorbedingung]
Die Testperson muss sich Registrieren und einloggen.
\item[Einzelschritte]~\\
Eingabe:
\begin{enumerate}
  \item Registrierung der Testperson am Webinterface.
  \begin{enumerate}
    \item Eingabe des Namens der Testperson.
    \item Eingabe der Email-Adresse.
    \item Eingabe des Passwortes.
  \end{enumerate}
  \item \label{Login} Login der Testperson am Webinterface.
  \begin{enumerate}
    \item Eingabe des Namens der Testperson.
    \item Eingabe des Passwortes.
  \end{enumerate}  
  \item Logout der Testperson am Webinterface.
	\item Passwort wiederherstellen.
	\begin{enumerate}
    \item Betätigen des "`Passwort vergessen"' Buttons.
    \item Eingabe des Namens, sowie der Email-Adresse.
		\item Betätigen des "`Neues Passwort generieren"' Buttons.
  \end{enumerate}		
  \item Erneut \ref{Login}, mit neuem generierten Passwort.
	\item Das Passwort ändern.
	\begin{enumerate}
    \item Betätigen des "`Passwort ändern"' Buttons.
    \item Eingabe des alten Passwortes, des neuen Passwortes, sowie Bestätigung des neuen Passwortes.
		\item Betätigen des "`Passwort ändern"' Buttons.
  \end{enumerate}
	\item Feeds hinzufügen.
	\begin{enumerate}
    \item \label{add} Die URL des Feeds in das Feld eingeben.
    \item Betätigen des "`Feed hinzufügen"' Buttons.
    \item Betrachten der, um den hinzugefügten Feed ergänzte, Feed-Liste. Evtl. erneut ab \ref{add}.
	\end{enumerate}
	\item Feeds löschen.
	\begin{enumerate}
    \item \label{delete} Haken in der Liste setzten, bei zu entfernender Feeds.
		\item Betätigen des "`Feeds entfernen"' Buttons.
		\item Bestätigen.
    \item Betrachten der um die gelöschten Feeds reduzierte Feed-Liste. Evtl. erneut ab \ref{delete}.
	\end{enumerate}
  \item Stellen einer Textanfrage an den \NewsGenie.
	\begin{enumerate}
	  \item \label{search} Anfrage in das Textfeld eingeben.
		\item Betätigen des "`Anfrage stellen"' Buttons.
    \item Manuelles Auswählen einer der vom \NewsGenie bereit gestellten Nachrichten.
		\item Evtl. erneut ab \ref{search}.
	\end{enumerate}
  \item Löschen des Accounts.
	\begin{enumerate}
	  \item Betätigen des "`Account löschen"' Buttons.
	  \item Bestätigen.
  \end{enumerate}
\end{enumerate}
Ausgabe:
Als Ausgabe erhalten wir eine Rückmeldung der Testperson bezüglich der Benutzerfreundlichkeit und Verständlichkeit
des Webinterfaces, sowie eine Liste mit allen evtl. aufgetretenen unerwarteten Reaktionen oder Fehlern.
\item[Beobachtungen / Log / Umgebung]~\\
Die Testperson wird bei der Bedienung beobachtet um so die Bedienbarkeit des Webinterfaces beurteilen zu können.
Dabei gilt es darauf zu achten, wie Intuitiv sich die Testperson durch das Webinterface klickt, und wieviele Fragen 
sie bezüglich der Bedienung hat.
\item[Besonderheiten]~\\
Es gibt keine Besonderheiten.
\item[Abhängigkeiten]~\\
Es gibt keine besonderen Abhängigkeiten.
\end{testcase}

\begin{testcase}{201}{Webinterface -- Administrator}
\item[Ziel]~\\
Es soll geprüft werden, ob ein Administrator eine Liste aller Benutzer angezeigt bekommt und 
dort Nutzer löschen, sowie auch deren Rolle übernehmen, kann.
\item[Objekte/Methoden/Funktionen]~\\
\ref{F90},\ref{F100}
\item[Pass/Fail Kriterien]~\\
Die dem Administrator angezeigte Liste aller Benutzer muss vollständig sein.
Nach dem löschen des Benutzers darf dieser nicht mehr aufgelistet sein.
Am Ende muss die Rolle zu demjenigen Benutzer gewechselt worden sein, dessen Rolle übernommen werden sollte.
\item[Vorbedingung]~\\
Der Administrator muss am Webinterface eingeloggt sein.
Es müssen bereits mindestens zwei weitere Accounts registriert worden sein.
Einer davon sollte extra für diesen Test, zum löschen, registriert worden sein.
\item[Einzelschritte]~\\
Eingabe:
\begin{enumerate}
\item Benutzer-Liste Anzeigen.
\begin{enumerate}
\item Den Knopf "`Administration"' drücken.
\end{enumerate}
Erwartete Beobachtung: Eine vollständige Liste aller Benutzer wird angezeigt.
\item Benutzer Löschen.
\begin{enumerate}
\item Das Auswahlkästchen neben dem zu löschenden Nutzer in der Liste anklicken.
\item Den Knopf "`Delete selected"' drücken.
\end{enumerate}
Erwartete Beobachtung: Der zuvor ausgewählte Nutzer wird nicht mehr aufgelistet.
\item Rolle wechseln.
\begin{enumerate}
\item Bei einem Benutzer in der liste den Knopf "`sudo"' drücken.
\end{enumerate}
Erwartete Beobachtung: Oben rechts, neben "`username:"' wird jetzt der Name des Nutzers angezeigt, dessen Rolle übernommen werden sollte.
\end{enumerate}
Ausgabe:
Eine Rückmeldung des Administrators, ob nach jedem Schritt, die erwarteten Beobachtungen eingetreten sind.
\item[Beobachtungen / Log / Umgebung]~\\
Der Administrator bekommt entweder direkten Zugriff auf die Datenbank, 
oder eine manuelle Liste aller real bereits registrierten Benutzer, um die Funktion \ref{F90} auf Vollständigkeit überprüfen zu können.
\item[Besonderheiten]~\\
Es gibt keine Besonderheiten.
\item[Abhängigkeiten]~\\
Der Testfall \ref{T201} ist Abhängig vom Testfall \ref{T200}. Das Registrieren, sowie das Einloggen muss bereits funktionieren (Siehe Vorbedingung).
\end{testcase}
