%!TEX root = ../Pflichtenheft.tex

\chapter{Glossar}

\begin{description}

\item[Account] Benutzerkonto, dient zur Identifizierung eines Benutzers und
speichert benutzerspezifische Daten.

\item[Administrator] Ein Benutzer mit besonderen Rechten zur Verwaltung von
anderen Benutzern und Ressourcen eines Systems.

\item[API] Programmierschnittstelle, wird von einem Programm zur Anbindung von
weiteren Programmteilen auf Quelltextebene zur Verfügung gestellt. 

\item[Backend] Für den Benutzer nicht sichtbarer Programmteil, läuft
typischerweise auf einem Server und stellt Daten für eine Clientapplikation zur
Verfügung.

\item[Client] Nutzer einer von einem Server über das Netzwerk angebotenen
Ressource.

\item[Crawler] Ein Serverdienst im Backend zur Erfassung und Verarbeitung von
Daten aus verschiedenen Quellen.

\item[Datenbank] Ein Programm zur strukturierten Datenhaltung mit definierter
Abfragesprache.

\item[Hash] Eine Prüfsumme, oft mit festgelegter Länge, die zur
eindeutigen Identifizierung von Datensätzen oder für kryptografische Zwecke
eingesetzt wird.

\item[Prozedur] Ein Verfahren mit festgelegter Befehlsreihenfolge zur
planmäßigen Lösung von Problemen. Bezeichnet in der Programmierung oft einen
Codeblock oder eine Funktion für eine bestimmte Aufgabe.

\item[Raspberry Pi] Günstiger Einplatinencomputer mit geringer
Leistungsaufnahme.

\item[RSS-Feed] Abkürzung für Really Simple Syndication-Feed, einfaches und
strukturiertes Format zur Veröffentlichung von Änderungen auf Webseiten. 
Ein RSS-Feed versorgt den Benutzer mit kurzen Informationsblöcken wie 
Nachrichtentitel und Textanriss, ähnlich einem Newsticker.

\item[Server] Ein Computer der zentrale Dienste in einem Netzwerk für mehrere
Nutzer zur Verfügung stellt. 

\item[String] Eine Folge von Zeichen (Zeichenkette) von definierten Symbolen und
fester oder variabler Länge.

\item[Query] bezeichnet eine Abfrage, oft an eine Datenbank oder
Serveranwendung, die häufig einen formalen Ausdruck erwartet.

\item[Query-Processor] Serverapplikation zur Beantwortung von Querys. Läuft im
Backend der Anwendung.

\item[Webinterface] Benutzerschnittstelle auf Basis einer Webseite, erlaubt die
Eingabe von Daten durch den Benutzer und gibt Rückmeldung über Erfolg/Misserfolg
der getätigten Einstellungen.

\end{description}
