%!TEX root = ../Systementwurf.tex

\chapter{Erfüllung der Kriterien}

In diesem Abschnitt wird beschrieben, wie die im Pflichtenheft aufgeführten Kriterien erfüllt
werden und worauf geachtet wird. Es wird zuerst auf die Musskriterien, dann auf die Soll und Kann-Kriterien und zuletzt auf die Abgrenzungskriterien eingegangen.


\section{Musskriterien}

Die folgenden Kriterien sind unabdingbar und müssen durch das Produkt erfüllt
werden:

\ref{RM1} Ein- und Ausgabe in natürlicher Sprache

Die Spracheingabe erfolgt durch ein am Raspberry Pi angebrachtes Mikrofon. Der \textit{Client}\ref{C10} nimmt gesprochene Sprache auf und wandelt diese mithilfe der Google-Speech-Api in Text um.
Auf dem Server wird der Text dann im \textit{Query-Processor}\ref{C50} mithilfe von Natural Language Proccessing und Named Entity Recognition analysiert und erkannt.
Die Sprachausgabe erfolgt durch einem Lautsprecher am Raspberry Pi. Der \textit{Client}\ref{C10} erhält auszugebendes als Text, wandelt es in Sprache um und gibt es anschließend aus.

\ref{RM2} Interaktives Verhalten

Durch drücken des Aufnahmeknopfes während einer Sprachausgabe, kann der Nutzer diese unterbrechen, um seine vorherige Suchanfrage noch weiter zu spezifizieren oder negatives Feedback (z.B. "`Stop"') zu geben.
Falls der \textit{Query-Processor}\ref{C50} eine Anfrage nicht klar verstehen konnte, kann der \textit{Client}\ref{C10} dem Nutzer eine Reihe von vordefinierten Ja/Nein Fragen stellen, um die Anfrage einzugrenzen.

\ref{RM3} Das System muss die englische Sprache unterstützen

\NewsGenie ist standardmäßig auf die englische Sprache eingestellt.
Jeder Nutzer kann am \textit{Webinterface}\ref{C20} einstellen, welche Sprache er verwenden möchte. 
Alle für die Lokalisierung verantwortlichen Inhalte werden in externe Textdateien verlagert, was das einbauen weiterer Sprachen stark vereinfacht.
Jeder Artikel in der \textit{Datenbank}\ref{C30} erhält einen Sprach-Tag, um zu
gewährleisten, dass jeder Nutzer nur Artikel seiner eingestellten Sprache entsprechend erhält.
Jegliche von uns verwendete Fremdanbietersoftware unterstützt auch standardmäßig die englische Sprache.

\pagebreak

\ref{RM4} Benutzerschnittstelle

Das \textit{Webinterface}\ref{C20} wird mit Hilfe des Play-Frameworks auf dem Server umgesetzt.

\ref{RM5} Backend

Artikel werden als Tripel in der \textit{Datenbank}\ref{C30} gespeichert und durch Lucene zur Suche aufbereitet.
Außerdem erhalten Artikel zusätzlich einen Zeitstempel um das Erkennen von veralteten Artikeln zu ermöglichten.

\section{Sollkriterien}

Die Erfüllung folgender Kriterien für das abzugebende Produkt wird angestrebt:

\ref{RS1} Entitäts-Erkennung

Mit dem Named Entity Recognizer aus Stanford NLP können Entitäten in gecrawlten Artikeln erkannt werden. 
Anschließend werden diese in der \textit{Datenbank}\ref{C30} abgelegt und durch die Tripelspeicherweise miteinander verknüpft. 
Dieses Verknüpfen ermöglicht es, Zusammenhänge zwischen Artikeln besser zu erkennen um so eine höhere Antwortqualität 
zu erreichen.

\ref{RS2} News Crawling

Der Crawler durchsucht regelmäßig die RSS-Feed-Quellen nach neuen Artikeln und speist diese anschließend in die \textit{Datenbank}\ref{C30} ein.
Neue Quellen können sowohl im \textit{Webinterface}\ref{C20} als auch per Spracheingabe am \textit{Client}\ref{C10} dynamisch dem Crawler hinzugefügt werden.

\ref{RS3} Deutsche Sprache

Die deutsche Sprache wird vollkommen unterstützt.
Auch die von uns verwendete Fremdanbietersoftware unterstützt die deutsche Sprache.

\ref{RS5} Persönlichkeit

Der \textit{Client}\ref{C10} kann rudimentäre Fragen wie z.B. {\glqq Hi, how are
you?\grqq} stellen, sowie auf einer kleinen Anzahl von Spaßfragen wie z.B.
{\glqq What does the Fox say?\grqq}  antworten.

\section{Kannkriterien}

Die Erfüllung folgender Kriterien für das abzugebende Produkt wird angestrebt:

\ref{RC1} Konkurrentes News-Crawling

Das Crawling findet auf dem Server in einem separatem Prozess statt, weswegen es die normalen Funktionen von \NewsGenie nicht beeinträchtigt.

\ref{RC2} Trend-Erkennung

Trends können auf zwei Arten gefunden werden:
Zum einem durch die Analyse, wie häufig welche Themen von Nutzern abgefragt wurden, 
und zum anderen beim crawlen, wenn zu einem Zeitpunkt bestimmte Themen besonders häufig in unterschiedlichen RSS-Feeds auftauchen.

\ref{RC3} Personalisierung

Zu jedem Nutzer werden in der \textit{Datenbank}\ref{C30} Nutzerdaten inklusive seiner Suchhistorie gespeichert.
Aus der Häufigkeit des Vorkommens von bestimmen Themen in der Suchhistorie eines Nutzers können dessen Interessen abgeleitet werden.

\section{Abgrenzungskriterien}

Folgende Funktionalitäten werden nicht durch das Produkt, sondern wie folgt
beschrieben anderweitig erfüllt

\ref{RW1} Das System soll die Sprachen nicht automatisch unterscheiden können

Der Nutzer kann am \textit{Webinterface}\ref{C20} die Sprache einstellen (Deutsch und Englisch).

\ref{RW2} Das System soll keine komplexen Fragen stellen

Zu komplexe Fragen mit vielen Nebensätzen sind äußerst schwer zu erkennen, was den Umfang dieses Praktikums 
übertreffen würde.
Des Weiteren genügen in den meisten Fällen einfache Fragen, um auszudrücken wonach man sucht.

\ref{RW3} Das System umfasst keine künstliche Intelligenz, Antworten werden
nicht umformuliert

Artikel werden in eine Auswahl von vordefinierten Sätzen eingepackt, aber sonst unverändert ausgeben, um dennoch natürlich wirken zu können.

\ref{RW4} Artikel, die nicht im RSS-Feed verlinkt sind, werden nicht
verarbeitet

Da fast jede Nachrichtenseite im Internet einen RSS-Feed anbietet, ist die Wahrscheinlichkeit groß, dass sich das gesuchte Thema in den dort verlinkenden Artikeln befindet.