%!TEX root = ../Systementwurf2.tex

% Kapitel 5
%-------------------------------------------------------------------------------
\chapter{Serverkonfiguration}

Der Server besteht insgesamt aus drei Komponenten, die jeweils einzeln gestartet und konfiguriert werden können. Der Server ist dazu konzipiert, auf einem Linux-System zu laufen.

\section{Virtuoso}

Zur Zeit der Auslieferung verwenden wir Virtuoso 6.1 in der opensource Variante. 
Virtuoso kann von \texttt{http://virtuoso.openlinksw.com/dataspace/doc/dav/wiki/Main/VOSDownload} heruntergeladen werden.

Die Konfigurationsdatei heißt \texttt{virtuoso.ini} und liegt standardmäßig im Verzeichnis \texttt{/etc/ $\hookleftarrow$""virtuoso-opensource-\%version\%/}.
In dieser Datei können insbesondere die Ports der Datenbank (standardmäßig 1111) und des Datenbankbackends (standardmäßig 8890) und der Ort, an dem die Virtuoso-Systemdateien liegen, (standardmäßig \texttt{/var/lib/virtuoso-opensource- $\hookleftarrow$""\%version\%/db/}) festgelegt werden.\\
Die ausführbare Datei heißt \texttt{virtuoso-t} und liegt in \texttt{/usr/bin/}. Zum Starten der Datenbank muss der Pfad der Konfiguration angegeben werden. Der Aufruf sieht dementsprechend folgendermaßen aus: \texttt{virtuoso-t [+foreground] +configfile /etc/virtuoso-opensource-~$\hookleftarrow$\linebreak\%version\%/virtuoso.ini}.

\section{Newsgenie-Server}

Der Newsgenie-Server ist eine Java-Anwendung und wird als ausführbares Jar-Archiv ausgeliefert. Es wird eine Java-Installation der Version 7 oder höher benötigt. Die Konfigurationsdatei heißt \texttt{server.cfg} und muss sich im gleichen Ordner wie die Jar-Datei befinden. In der Konfiguration können die Server-IP-Adresse sowie der Akka- und der Virtuoso-Port festgelegt werden. Die Server-IP muss mit der wirklichen IP des Servers übereinstimmen und der Virtuoso-Port muss dem Port entsprechen, auf dem Virtuoso Anfragen entgegennimmt (siehe oben).\\
Der Server kann mittels \texttt{java -jar server.jar} gestartet werden. Hierzu ist es notwendig, dass die Datenbank bereits läuft und erreichbar ist.

\section{Webinterface}

Das Webinterface ist eine Play-Anwendung. Das Play-Framework kann von  \texttt{http://www.play~$\hookleftarrow$\linebreak framework.com/download} heruntergeladen werden. Danach kann das Webinterface mittels \texttt{play run} gestartet werden. Um das Webinterface auszuführen, müssen sowohl Datenbank als auch Server laufen. Die Konfigurationsdatei heißt \texttt{application.conf} und liegt im Unterordner \texttt{./conf/}. Hier kann zum Beispiel die IP-Adresse geändert werden.