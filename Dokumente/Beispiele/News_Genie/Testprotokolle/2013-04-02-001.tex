%!TEX root = ../Testprotokolle.tex

\chapter{Testdurchführung (2012-04-02)}

In diesem Abschnitt werden die einzelnen Durchführungen (=Testläufe) protokolliert.
Der Testlauf beschreibt eine Durchführung eines Testfalls. Derselbe
Testfall kann mit verschiedenen Eingabedaten oder auch mit verschiedenen
Softwareversionen mehrmals durchgeführt werden. Für jeden Testlauf wird diese .tex-Datei kopiert\\

Zunächst wird eine Aufstellung der durchgeführten Testfälle, der abgedeckten Funktionen, Komponenten, Klassen und Methoden sowie der beteiligten Test gegeben.

Beispiel:\\
Art des Tests: Abnahmetest\\
Ausgeführte Testfälle: \ref{T100}, \ref{T200}\\ 
Beteiligte Tester: Max Mustermann\\
Abgedeckte Funktionen: \ref{F10}, \ref{F20}

\section{Testumgebung}

Kurze Beschreibung der Testumgebung in der die Tests durchgeführt wurden, so dass eine Reproduktion der Ergebnisse möglich ist.

Beispiel\\
Die Testfälle wurden unter Windows 7 auf einem lokalen Webserver (Apache Tomcat 7.0.39) durchgeführt. Es wurde eine deutsche Systemumgebung verwendet. Die Seiten wurden mit Firefox 19.0.2 geöffnet.

\section{Testprotokoll}

Die folgenden Tabellen beschreiben, wie der Testfall ausgeführt wurde und
welches Ergebnis er geliefert hat. Da es bei Korrektur von Softwarefehlern oder
anderen Gegebenheiten notwendig ist, einen Test mehrfach durchzuführen
(Testläufe), ist jede Testdurchführung zu dokumentieren. Daher ist diese
Tabelle für \textbf{jeden Testlauf} zu erstellen und \textbf{fortlaufend zu
nummerieren}. \\

\begin{longtable}{|p{4cm}|p{11cm}|}
\hline
\textbf{Testfall} & \textit{Beispiel: \ref{T100}} \\
\hline
\textbf{Tester} & \textit{Beispiel: Max Mustermann} \\
\hline
\textbf{Eingaben} & \textit{Es sind alle Eingabedaten bzw. andere Aktionen
aufzufüh-ren, die für die Ausführung des Testfalls notwendig sind.
Diese können sowohl als Wert angegeben werden (ggf. mit Toleranzen) als auch
als Name, falls es sich um konstante Tabellen oder um Dateien handelt. Außerdem
sind alle betroffenen Datenbanken, Dateien, Terminal Meldungen, etc. anzugeben.
Hinweis: Es sind nicht noch mal die Einzelschritte aus 3.1.3 zu wiederholen.
Während jene allgemeiner sind (z.B. "`Ein-loggen über das Login-Formular"')
sind hier die konkreten eingegebenen Testdaten aufzuführen (z.B. "`Loginname:
test; Passwort: xxxtest"'`). } \\
\hline
\textbf{Soll - Reaktion} & \textit{Hier ist anzugeben, welches Ergebnis bzw.
Ausgabe der Test haben soll.
Hinweis: Es sind nicht noch mal die Erfolgskriterien aus 3.1.2 zu wiederholen.
Während jene allgemeiner sind (z.B. "`Testnachricht wird über Netzwerkkanal
empfangen"') sind hier die konkreten erhaltenen Testdaten aufzuführen (z.B.
Konsole zeigt Meldung: "`Testnachricht 123 erhalten"').
} \\
\hline
\textbf{Ist -- Reaktion} & \textit{Hier ist anzugeben, welches Ergebnisdaten
bzw. Ausgaben dieser Testlauf geliefert hat.} \\
\hline
\textbf{Ergebnis} & \textit{Für jeden Testlauf ist zu vermerken, ob der Test
erfolgreich durchgeführt werden konnte oder nicht. Einen missglückten Test
bitte begründen, sofern der Grund des Fehlschlags bekannt oder offensichtlich
ist.} \\
\hline
\textbf{Unvorhergesehene Ereignisse} &
\textit{optional; nur anzugeben, falls es unvorhergesehene Ereignisse gab} \\
\hline
\textbf{Nacharbeiten } & \textit{Ist ein Testlauf nicht erfolgreich
durchgeführt worden, so werden hier die erforderlichen Nacharbeiten aufgeführt
(z.B. Bugfixes).} \\
\hline
\end{longtable}

\section{Zusammenfassung}

Hier wird eine Zusammenfassung der Testergebnisse aufgelistet. Dabei sind alle
behandelten Probleme aufzulisten und darzustellen, wie ihre Lösungen erreicht
wurden. Dabei kann auf folgendes eingegangen werden:
\begin{itemize}
\item Zusammenfassung aller Hauptaktivitäten und der wichtigsten Ereignisse und Ergebnisse der Tests
\item Abweichungen der vorliegenden Software von der Aufgabenstellung. Aber auch Abweichung vom Testplan oder den Testfällen während des Testens (z.B.
   aufgrund von veränderter Funktionalität während der Nacharbeiten). Die
   Abweichungen sollten begründet werden.
\item Umfang des Testverlaufs (Vollständigkeit) im Vergleich zu Umfangskriterien
   des Testplans. Auflistung der Funktionalitäten, die nicht getestet wurden.
   Selbstversändlich mit Begründung.
\item Bewertung der Softwarequalität
\end{itemize}
