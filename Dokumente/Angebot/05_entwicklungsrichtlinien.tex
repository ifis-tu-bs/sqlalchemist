%!TEX root = ../Angebot.tex

\chapter{Entwicklungsrichtlinien}

\section{Konfigurationsmanagement}

Es gibt ein zentrales Repository auf Basis von Subversion (SVN), welches \"uber das Redmine des Instituts f\"ur Softwaretechnik und Fahrzeuginformatik (ISF) zur Verf\"ugung gestellt wird. Dieses Repository ist in drei Ordner unterteilt. In einem Ordner werden alle im Laufe des Projektes zu angefertigten Dokumente, die die Projektorganisation und den Kunden betreffen, abgelegt. Dazu z\"ahlen: ein Fachentwurf, eine Abnahmespezifikation, ein Pflichtenheft, ein technischer Entwurf, eine Testspezifikation und die entsprechenden Testprotokolle. Au{\ss}erdem gibt es zwei weitere Ordner, in die alle Dateien der jeweiligen Teilprojekte, das Back- sowie das Front-End, abgelegt werden. Hierbei handelt es sich unter anderem um Dateien mit dem Quellcode der jeweiligen Software-Komponente.  

Des Weiteren gibt es eine feste Kommentarstruktur, die bei jedem SVN-Commit, also dem Hochladen der ge\"anderten Dateien in das SVN-Repository, eingehalten werden soll. Dies f\"uhrt dazu, dass alle \"Anderungen nachvollziehbar, verst\"andlich und vollst\"andig sind. Um die Integrit\"at der Applikation zu gew\"ahrleisten, soll ebenfalls nur vollst\"andig ausf\"uhrbarer Code in das Repository hochgeladen werden.  


\section{Design- und Programmierrichtlinien}

F\"ur die Entwicklung des gesamten Projektes ist festgelegt, dass sich die Entwickler an folgende im Rahmen der Veranstaltung \glqq Software-Engineering I (SE I)\grqq~vorgestellten Coding-Guidelines halten:
\begin{itemize}
	\item Einr\"uckung: Um den Code \"ubersichtlicher zu gestalten, wird die Einr\"uckungstiefe auf 4 Zeichen festgelegt.
	\item Klammersetzung: Jede Klammer die zur Schlie{\ss}ung einer Methode bzw. einer Klasse gebraucht wird, wird in eine neue Zeile gesetzt. 
	\item Bezeichner: S\"amtliche Bezeichener (Parameter, Variablennnamen, Methodennamen, Klassennamen, etc.) werden sinnvoll nach ihrem Inhalt 		bennant, damit direkt ersichtlich ist, was die jeweilige Methode berechnet oder welche Information in jeder Variable steht.
	\item CamelCaseNotation: Zur besseren Lesbarkeit des Quelltextes werden Methoden in CamelCaseNotation notiert. 
	\item UPPER\_CASE Notation: Alle Konstanten werden in Gro{\ss}buchstaben, durch Unterstriche getrennt, geschrieben.
	\item lower\_case Notation: Imports und Packages werden in Kleinbuchstaben geschrieben.	
\end{itemize} 

Dar\"uber hinaus ist entschieden, s\"amtliche Kommentare auf Englisch zu verfassen.
Das Software-Dokumentationswerkzeug Javadoc generiert parallel aus dem Java-Quellcode eine vollst\"andige HTML-Dokumentationsdatei.

%!(Bestandteil des Java Development Kits) glossar


\section{Verwendete Software}

Als Entwicklungsumgebung wird IntelliJ IDEA von JetBrains verwendet, zum Erstellen von Diagrammen und technischen Zeichnungen  das Visualisierungsprogramm Dia, sowie Microsoft Visio. Textdokumente, die das Projekt betreffen, werden mit Hilfe des Software-Pakets LaTeX angefertigt und formatiert.

Das Front-End wird mithilfe der HTML5 Spiele-Engine melonJS und der plattformunabh\"angigen JavaScript-Bibliothek jQuery entwickelt, das Admin-Tool des Back-Ends mit dem Open-Source-Framework AngularJS und dem CSS-Framework Bootstrap.
F\"ur die Server-Komponente des Back-Ends wird das Web-Framework play! verwendet. 

Das Front-End-Team nutzt f\"ur das zu entwickelnde Mini-Spiel den Tiled Map Editor. Grafiken f\"ur die gesamte Software werden mit Hilfe des Bildbearbeitungsprogramm Adobe Photoshop CS6 entstehen.

Au{\ss}erdem wird eine Software des mitarbeitenden Teamprojektes, die automatisch Schemata generiert zur Verf\"ugung gestellt. Zudem kann auf eine Software, die im Rahmen einer Bachelorarbeit entwickelt wird, zur\"uckgegriffen werden, die sich mit der automatischen Generierung von SQL-Statements und Aufgabenstellungen besch\"aftigt.



