%!TEX root = ../Angebot.tex

\chapter{Einleitung}

Im Folgenden wird zuerst das Motiv f\"ur das Projekt \glqq Das Gro\ss{}e SQL-Spiel\grqq~erl\"autert und anschlie{\ss}end werden Projektablauf, -umfang und -organisation sowie s\"amtliche technische Grundlagen als auch Richtlinien gekl\"art. Dabei wird grunds\"atzlich zwischen einem Back- sowie einem Front-End unterschieden. Das Back-End steht hierbei f\"ur die Serverkomponente der Software. Das Front-End beschreibt die gesamte Oberfl\"ache und Visualisierung der Software.

%!\textbf{Hinweis zu den Templates:}

\section{Ziel}

Das Ziel ist es, eine plattform\"ubergreifende, interaktive Spielesoftware zu entwickeln, um den Studenten der Pflichtveranstaltung \glqq Relationale Datenbanksysteme I (RDBI)\grqq~den Umgang mit der Datenbankanfragesprache SQL vorlesungsbegleitend spielerisch zu vermitteln. Sie soll ab dem Wintersemester 2015 vollst\"andig in den \"Ubungsbetrieb der Vorlesung integriert werden. 

Es soll eine M\"oglichkeit geschaffen werden, praktische Aspekte im Bereich der Datenbankanfragen zu \"uben, da dies im Rahmen einer Lehrveranstaltung viele verschiedene Schwierigkeiten mit sich bringt. Zum Einen kann theoretisches Verst\"andnis zwar gut vermittelt werden, jedoch ist es nur eingeschr\"ankt m\"oglich, gen\"ugend \"Ubungsmaterial zur Verf\"ugung zu stellen, damit eine komplexe Anfragesprache, wie SQL, ge\"ubt und somit auch verstanden werden kann.

Die Software soll genau diese M\"angel aufgreifen und beheben.


\section{Motivation} 

SQL hat die F\"ahigkeit komplexe, schwer nachvollziehbare Anfragen an eine Datenbank zu stellen. In der Theorie wird durchaus verstanden was hinter einer Datenbankanfrage steht, jedoch sind die Studenten meistens nicht in der Lage dies z\"ugig in der Sprache SQL auszudr\"ucken. Damit dieses verstanden werden kann, muss es permanent wiederholt und ge\"ubt werden. 
Zur \"Ubung von SQL sind jedoch kaum sinnvolle Tools vorhanden, die ohne gro{\ss}en Aufwand benutzt werden k\"onnen. 
Der Aufwand besteht darin, einen Datenbankserver zu installieren, zu konfigurieren, sich entsprechende Datenmodelle zu \"uberlegen und diese mit sinnvollen Daten zu verkn\"upfen.   
Es soll eine Software entwickelt werden, die die M\"oglichkeit bietet, dieses den Studenten zu erleichtern. Es wird eine komplett funktionsf\"ahige Applikation erstellt, die unkompliziert aus jedem Webbrowser aufrufbar sein wird.

Nun bietet die zu entwickelnde Software genau diese M\"oglichkeiten. Wie ist es dann aber m\"oglich die Studenten zu motivieren sich mit genau diesem Programm auseinander zu setzten um tats\"achlich SQL zu \"uben? Hier greift das Prinzip der \glqq Gamification\grqq. Mit Hilfe von spielerischen Aspekten in Form eines Mini-Spiels und t\"aglichen bzw. w\"ochentlichen Challenges sollen die Studenten motiviert werden. Durch immer wechselnde, zuf\"allig generierte Herausforderungen und damit verbundenen Erfolgserlebnissen sollen die User belohnt werden und somit zum regelm\"a{\ss}igen Gebrauch dieser App animiert werden.
