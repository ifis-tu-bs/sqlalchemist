%!TEX root = ../Testprotokolle.tex

\chapter{Testdurchführung (2015-07-10)}

Art des Tests: Integrationstest\\
Testfall: T700\\
Abgedeckte Funktionen: \textbf{F20}
Beteiligte Tester: Denis Nagel\\

\section{Testumgebung}

Die Funktion wurde unter Windows 8.1 auf einem Webserver  getestet. Es wurde eine deutsche Systemumgebung verwendet. Die Applikation wurde über Firefox 39.0 gestartet.

\section{Testprotokoll}

\begin{longtable}{|p{4cm}|p{11cm}|}
\hline
\textbf{Funktion} & \textit{\textbf{F20}} \\
\hline
\textbf{Tester} & \textit{Denis Nagel} \\
\hline
\textbf{Eingaben} & \textit{Username: Testnutzer, E-Mail-Adresse: testnutzer@local.de, Passwort: password1234} \\
\hline
\textbf{Soll - Reaktion} & \textit{Die eingegebenen Daten werden erkannt und der Nutzer wird angemeldet, sowie zum Hauptmenü weitergeleitet.} \\
\hline
\textbf{Ist -- Reaktion} & \textit{Der Nutzer wurde erfolgreich angemeldet und ins Hauptmenü weitergeleitet.} \\
\hline
\textbf{Ergebnis} & \textit{Der Testlauf ist erfolgreich abgeschlossen worden.} \\
\hline
\textbf{Unvorhergesehene Ereignisse} &
\textit{} \\
\hline
\textbf{Nacharbeiten } & \textit{} \\
\hline
\end{longtable}

\newpage
\begin{longtable}{|p{4cm}|p{11cm}|}
\hline
\textbf{Funktion} & \textit{\textbf{F20}} \\
\hline
\textbf{Tester} & \textit{Denis Nagel} \\
\hline
\textbf{Eingaben} & \textit{Username: FalscherNutzer, E-Mail-Adresse: testnutzer@local.de, Passwort: password1234} \\
\hline
\textbf{Soll - Reaktion} & \textit{Es wird dem Nutzer mitgeteilt, dass die eingegebenen Daten nicht korrekt sind.} \\
\hline
\textbf{Ist -- Reaktion} & \textit{Die korrekte Meldung wurde ausgegeben.} \\
\hline
\textbf{Ergebnis} & \textit{Der Testlauf ist erfolgreich abgeschlossen worden.} \\
\hline
\textbf{Unvorhergesehene Ereignisse} &
\textit{} \\
\hline
\textbf{Nacharbeiten } & \textit{} \\
\hline
\end{longtable}

\newpage
\begin{longtable}{|p{4cm}|p{11cm}|}
\hline
\textbf{Funktion} & \textit{\textbf{F20}} \\
\hline
\textbf{Tester} & \textit{Denis Nagel} \\
\hline
\textbf{Eingaben} & \textit{Username: Testnutzer, E-Mail-Adresse: falschernutzer@local.de, Passwort: password1234} \\
\hline
\textbf{Soll - Reaktion} & \textit{Es wird dem Nutzer mitgeteilt, dass die eingegebenen Daten nicht korrekt sind.} \\
\hline
\textbf{Ist -- Reaktion} & \textit{Die korrekte Meldung wurde ausgegeben.} \\
\hline
\textbf{Ergebnis} & \textit{Der Testlauf ist erfolgreich abgeschlossen worden.} \\
\hline
\textbf{Unvorhergesehene Ereignisse} &
\textit{} \\
\hline
\textbf{Nacharbeiten } & \textit{} \\
\hline
\end{longtable}

\newpage
\begin{longtable}{|p{4cm}|p{11cm}|}
\hline
\textbf{Funktion} & \textit{\textbf{F20}} \\
\hline
\textbf{Tester} & \textit{Denis Nagel} \\
\hline
\textbf{Eingaben} & \textit{Username: Testnutzer, E-Mail-Adresse: testnutzer@local.de, Passwort: wrongpassword} \\
\hline
\textbf{Soll - Reaktion} & \textit{Es wird dem Nutzer mitgeteilt, dass die eingegebenen Daten nicht korrekt sind.} \\
\hline
\textbf{Ist -- Reaktion} & \textit{Die korrekte Meldung wurde ausgegeben.} \\
\hline
\textbf{Ergebnis} & \textit{Der Testlauf ist erfolgreich abgeschlossen worden.} \\
\hline
\textbf{Unvorhergesehene Ereignisse} &
\textit{} \\
\hline
\textbf{Nacharbeiten } & \textit{} \\
\hline
\end{longtable}

\section{Zusammenfassung}

Der Testlauf hat gezeigt, dass die Anmeldung von Nutzern wie gewünscht funktioniert.

