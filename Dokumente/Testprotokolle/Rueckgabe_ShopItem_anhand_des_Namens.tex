%!TEX root = ../Testprotokolle.tex

\chapter{Testdurchführung (2012-04-02)}



\begin{longtable}{|p{4cm}|p{11cm}|}
\hline
\textbf{Testfall} & \textit{Rückgabe von ShopItems anhand der ID \textbf{T100}} \\
\hline
\textbf{Tester} & \textit{Stefan Hanisch} \\
\hline
\textbf{Eingaben} & \textit{@Before \newline
public void app() \{ \newline
\hspace*{1mm}FakeApplication fakeApplication \newline
\hspace*{4mm}= fakeApplication(inMemoryDatabase("'test"'));\newline
\hspace*{1mm}start(fakeApplication); \newline
\} \newline
\newline
@Test \newline
public void testShopItemGetByName() \{\newline
\hspace*{1mm}boolean testSuccessful = true; \newline
\hspace*{1mm}try \{\newline
\hspace*{3mm}ShopItem shopItem = ShopItem.getByName("'Alastor Moody"');\newline
\hspace*{3mm}assertEquals(shopItem.getName(), "'Alastor Moody"');\newline
\hspace*{1mm}\} catch (Throwable e) \{ \newline
\hspace*{3mm}testSuccessful = false; \newline 
\hspace*{1mm}\} \newline
\hspace*{1mm}assertTrue(testSuccessful);\newline
\hspace*{1mm}Logger.info("'ShopItemGetByName successfull"');\newline
\}
} \\
\hline
\textbf{Soll - Reaktion} & \textit{Konsole zeigt Meldung: "'ShopItemGetByName successfull"'
} \\
\hline
\textbf{Ist -- Reaktion} & \textit{Konsole zeigt Meldung: "'ShopItemGetByName successfull"'} \\
\hline
\textbf{Ergebnis} & \textit{Test erfolgreich durchgeführt} \\
\hline
\textbf{Unvorhergesehene Ereignisse} &
\textit{--} \\
\hline
\textbf{Nacharbeiten } & \textit{--} \\
\hline
\end{longtable}
