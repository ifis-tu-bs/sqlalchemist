%!TEX root = ../Testprotokolle.tex
\chapter{Testdurchführung (2015-07-13)}

Art des Tests: Integrationstest\\
Testfall: T800\\
Abgedeckte Funktionen: \textbf{F130}
Beteiligte Tester: Fabio Mazzone \& Gabriel Ahlers\\

\section{Testumgebung}

Die Funktion wurde unter Ubuntu 14.04 LTS auf einem Webserver  getestet. Es wurde eine deutsche Systemumgebung verwendet. Die Applikation wurde im Chromium Web Browser (Version: 43.0.2357.81) gestartet.

\section{Testprotokoll}

\begin{longtable}{|p{4cm}|p{11cm}|}
\hline
\textbf{Funktion} & \textit{\textbf{F130}} \\
\hline
\textbf{Tester} & \textit{Fabio Mazzone \& Gabriel Ahlers} \\
\hline
\textbf{Eingaben} & \textit{Im SQL-Trainer (Funktionsweise in allen Modi gleich) gibt der Benutzer ein korrektes Statement als Antwort auf eine Aufgabe} \\
\hline
\textbf{Soll - Reaktion} & \textit{Dem Front-End soll mitgeteilt werden, dass die Lösung korrekt ist. Außerdem soll im Profil gespeichert werden, dass die Aufgabe gelöst wurde} \\
\hline
\textbf{Ist -- Reaktion} & \textit{Es wurde übergeben, dass das die Lösung korrekt ist. Die Markierung wurde ebenfalls angelegt.} \\
\hline
\textbf{Ergebnis} & \textit{Der Testlauf ist erfolgreich abgeschlossen worden.} \\
\hline
\textbf{Unvorhergesehene Ereignisse} &
\textit{} \\
\hline
\textbf{Nacharbeiten } & \textit{} \\
\hline
\end{longtable}

\newpage
\begin{longtable}{|p{4cm}|p{11cm}|}
\hline
\textbf{Funktion} & \textit{\textbf{F130}} \\
\hline
\textbf{Tester} & \textit{Fabio Mazzone \& Gabriel Ahlers} \\
\hline
\textbf{Eingaben} & \textit{Im SQL-Trainer (Funktionsweise in allen Modi gleich) gibt der Benutzer ein leeres Statement als Antwort auf eine Aufgabe} \\
\hline
\textbf{Soll - Reaktion} & \textit{Dem Front-End soll mitgeteilt werden, dass die Lösung nicht korrekt ist.} \\
\hline
\textbf{Ist -- Reaktion} & \textit{Es wurde übergeben, dass die Lösung nicht korrekt ist.} \\
\hline
\textbf{Ergebnis} & \textit{Der Testlauf ist erfolgreich abgeschlossen worden.} \\
\hline
\textbf{Unvorhergesehene Ereignisse} &
\textit{} \\
\hline
\textbf{Nacharbeiten } & \textit{} \\
\hline
\end{longtable}

\newpage
\begin{longtable}{|p{4cm}|p{11cm}|}
\hline
\textbf{Funktion} & \textit{\textbf{F130}} \\
\hline
\textbf{Tester} & \textit{Fabio Mazzone \& Gabriel Ahlers} \\
\hline
\textbf{Eingaben} & \textit{Im SQL-Trainer (Funktionsweise in allen Modi gleich) gibt der Benutzer ein leeres Statement als Antwort auf eine Aufgabe} \\
\hline
\textbf{Soll - Reaktion} & \textit{Der Server soll die Anfrage sofort verwerfen und dies dem Front-End mitteilen.} \\
\hline
\textbf{Ist -- Reaktion} & \textit{Das Front-End überprüft bereits, ob die Antwort leer ist, somit erreicht die Anfrage den Server nicht und die Funktion wird nicht aufgerufen.} \\
\hline
\textbf{Ergebnis} & \textit{Der Testlauf ist erfolgreich abgeschlossen worden.} \\
\hline
\textbf{Unvorhergesehene Ereignisse} &
\textit{} \\
\hline
\textbf{Nacharbeiten } & \textit{} \\
\hline
\end{longtable}

\section{Zusammenfassung}

Der Testlauf hat gezeigt, dass die Funktionen des SQL-Trainers wie gewünscht funktionieren.
