%!TEX root = ../Testprotokolle.tex

\chapter{Testdurchführung (2012-04-02)}

In diesem Abschnitt werden die einzelnen Durchführungen (=Testläufe) protokolliert.
Der Testlauf beschreibt eine Durchführung eines Testfalls. Derselbe
Testfall kann mit verschiedenen Eingabedaten oder auch mit verschiedenen
Softwareversionen mehrmals durchgeführt werden. Für jeden Testlauf wird diese .tex-Datei kopiert\\

Zunächst wird eine Aufstellung der durchgeführten Testfälle, der abgedeckten Funktionen, Komponenten, Klassen und Methoden sowie der beteiligten Test gegeben.

Beispiel:\\
Art des Tests: Abnahmetest\\
Ausgeführte Testfälle: \textbf{T100}, \textbf{T200}\\ 
Beteiligte Tester: Max Mustermann\\
Abgedeckte Funktionen: \textbf{F10}, \textbf{F20}

\section{Testumgebung}



\section{Testprotokoll}

\begin{longtable}{|p{4cm}|p{11cm}|}
\hline
\textbf{Testfall} & \textit{\textbf{T180}} \\
\hline
\textbf{Tester} & \textit{Sören van der Wall} \\
\hline
\textbf{Eingaben} & \textit{Öffnen der entsprechenden Seite im Admin-Tool.} \\
\hline
\textbf{Soll - Reaktion} & \textit{Alle Ergebnisse der als Studenten registrierten User werden angezeigt.
} \\
\hline
\textbf{Ist -- Reaktion} & \textit{Nichts} \\
\hline
\textbf{Ergebnis} & \textit{Fehlschlag} \\
\hline
\textbf{Unvorhergesehene Ereignisse} &
\textit{optional; nur anzugeben, falls es unvorhergesehene Ereignisse gab} \\
\hline
\textbf{Nacharbeiten } & \textit{Class HomeWorkController und Class HomeWorkChallenge werden fertiggestellt. Zudem erhällt die Class User die Funktionalität User auf Studenten zu überprüfen.} \\
\hline
\end{longtable}

\begin{longtable}{|p{4cm}|p{11cm}|}
\hline
\textbf{Testfall} & \textit{\textbf{T190}} \\
\hline
\textbf{Tester} & \textit{Sören van der Wall} \\
\hline
\textbf{Eingaben} & \textit{Keinerlei} \\
\hline
\textbf{Soll - Reaktion} & \textit{User wird befördert.
} \\
\hline
\textbf{Ist -- Reaktion} & \textit{Keine} \\
\hline
\textbf{Ergebnis} & \textit{Erfolgreich getestet.} \\
\hline
\textbf{Unvorhergesehene Ereignisse} &
\textit{Die Funktion wurde beabsichtigt entfernt. Stattdessen geschieht die Beförderung nach Absprache mit dem Kunden nun automatisch nach einer vereinbarten Menge gelöster Aufgaben.} \\
\hline
\textbf{Nacharbeiten } & \textit{Nichts.} \\
\hline
\end{longtable}

\begin{longtable}{|p{4cm}|p{11cm}|}
\hline
\textbf{Testfall} & \textit{\textbf{T210}} \\
\hline
\textbf{Tester} & \textit{Sören van der Wall} \\
\hline
\textbf{Eingaben} & \textit{XML-Schema zur Aufgabenerstellung} \\
\hline
\textbf{Soll - Reaktion} & \textit{Pop-Up schließt, Aufgabe wird im Back-End erstellt.
} \\
\hline
\textbf{Ist -- Reaktion} & \textit{Pop-Up schließt, Aufgabe wird im Back-End erstellt.} \\
\hline
\textbf{Ergebnis} & \textit{Erfolgreich getestet.} \\
\hline
\textbf{Unvorhergesehene Ereignisse} &
\textit{optional; nur anzugeben, falls es unvorhergesehene Ereignisse gab} \\
\hline
\textbf{Nacharbeiten } & \textit{Nichts.} \\
\hline
\end{longtable}

\begin{longtable}{|p{4cm}|p{11cm}|}
\hline
\textbf{Testfall} & \textit{\textbf{T220}} \\
\hline
\textbf{Tester} & \textit{Sören van der Wall} \\
\hline
\textbf{Eingaben} & \textit{Klick auf einen der bereitstehenden Symbole zum Bewerten} \\
\hline
\textbf{Soll - Reaktion} & \textit{Back-End erfährt die Wertänderung, führt diese durch. Das Admin-Tool erfasst die Änderung.
} \\
\hline
\textbf{Ist -- Reaktion} & \textit{Wie erwartet.} \\
\hline
\textbf{Ergebnis} & \textit{Erfolgreich getestet.} \\
\hline
\textbf{Unvorhergesehene Ereignisse} &
\textit{optional; nur anzugeben, falls es unvorhergesehene Ereignisse gab} \\
\hline
\textbf{Nacharbeiten } & \textit{Nichts.} \\
\hline
\end{longtable}

\begin{longtable}{|p{4cm}|p{11cm}|}
\hline
\textbf{Testfall} & \textit{\textbf{T230}} \\
\hline
\textbf{Tester} & \textit{Sören van der Wall} \\
\hline
\textbf{Eingaben} & \textit{Auswählen der TaskFiles für die Hausaufgaben. Dann klick auf "Make HomeWork". Eingabe eines Namens und des gewünschten Zeitraums} \\
\hline
\textbf{Soll - Reaktion} & \textit{Hausaufgabe wird im BE erstellt. Admin-Tool erfasst die neue Homework. 
} \\
\hline
\textbf{Ist -- Reaktion} & \textit{Hausaufgabe wird nicht gespeichert.} \\
\hline
\textbf{Ergebnis} & \textit{Die entsprechende Funktionalität muss noch im Back-End eingefügt werden.} \\
\hline
\textbf{Unvorhergesehene Ereignisse} &
\textit{optional; nur anzugeben, falls es unvorhergesehene Ereignisse gab} \\
\hline
\textbf{Nacharbeiten } & \textit{Class HomeWorkController und Class HomeWorkChallenge werden im Back-End noch fertiggestellt.} \\
\hline
\end{longtable}

\section{Zusammenfassung}

?!?

\begin{itemize}
\item 
\item 
\item 
\item 
\end{itemize}
