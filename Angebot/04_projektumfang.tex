%!TEX root = ../Angebot.tex

% Kapitel 4
%-------------------------------------------------------------------------------

\chapter{Projektumfang}

\section{Lieferumfang}

Zum Lieferumfang der Software geh\"oren:

\begin{itemize}
	\item der Quellcode des Programms
	\item die ausf\"uhrbare Software
	\item die hinter der Nutzerschnittstelle liegende Datenbank
	\item das zugeh\"origes Handbuch
	\item die Spezifikation und der Entwurf der Software
	\item die Protokolle der Tests
	
\end{itemize}

\section{Kostenplan}

Die f\"ur das Projekt anfallenden Kosten, werden auf ca. 316.000 Euro gesch\"atzt. Davon entfallen
315.000 Euro auf den Lohn der Projektmitglieder (ausgehend von einem Arbeitslohn von 100 Euro pro Stunde und 30 Arbeitsstunden je Mitglied) 
und 1000 Euro auf den Kauf von Sprites, vorgefertigten Spieleger\"usten und weiteren f\"ur das Spiel ben\"otigten Objekten.

\section{Funktionaler Umfang}

Folgende Funktionen sollen in der abschlie{\ss}enden Version der Software enthalten sein:
\begin{enumerate}
	\item Insgesamt soll das Spiel dem Nutzer 3, beziehungsweise 4 verschiedene Modi bieten:
	\begin{enumerate}
      		\item Einen narrativen Modus, welcher dem Spieler erm\"oglicht in der Rolle des Alchemisten verschiedene Aufgaben zu 
			 erf\"ullen, um dann in der Geschichte des Spiels voranzuschreiten. Wobei es unabdingbar ist, dass der User
			 SQL-Anfragen \"ubt um in dem Modus weiter zu kommen.
		\item Einen Trivia-Modus, durch den die Spieler abseits eines festgelegten Handlungsstranges SQL-Abfragen \"uben 
			 k\"onnen,	indem sie zuf\"allig gestellte Aufgaben l\"osen.
		\item Einen Sandbox-Modus, der den Trivia-Modus durch vom Nutzer selbst erstellte Aufgaben erweitert. Diese Aufgaben 
			 stehen dann allen Nutzern zur Verf\"ugung und k\"onnen bewertet werden. Das Erstellen von Aufgaben jedoch bleibt freigeschalteten Spielern 
			 vorenthalten die sich im narrativen Modus bew\"ahrt haben. Dies wir durch das Back-End mit bestimmter Rechteverwaltung
			 realisiert.
		\item Einen Season-Modus, welcher von den Dozenten des Studienfaches RDBI genutzt werden kann, um die Software in den 
			 \"Ubungsbetrieb zu integrieren.
	\end{enumerate}
	\item Nutzerprofile und -statistiken, die den Fortschritt und die Erfolge eines jeden Spielers speichern. Au{\ss}erdem sollen diese grafisch
		aufbereitet werden und von jedem User abrufbar sein.
	\item Ranglisten, \"uber die sich die Nutzer untereinander vergleichen k\"onnen um die Motivation zu f\"ordern. 
		 Eventuell ist die Nutzung angemeldeten Spielern vorbehalten. 
	\item Ein Minispiel, welches sowohl im Story-, als auch im Trivia-Modus zum Einsatz kommt. Dabei handelt es sich um ein 
		 \glqq Jump \& Run\grqq-Spiel, in welchem die Spieler Gegenst\"ande einsammeln um Punkte zu erhalten und dabei 
		 SQL-Hindernisse \"uberwinden m\"ussen.
	\item Eine Benutzerverwaltung, die einzelne Rechte, sowie Rollen verwalten soll.
	\item Ein Admin-Tool, damit Dozenten und deren Mitarbeiter die Software in 
   		 ihre \"Ubungen integrieren k\"onnen. Insbesondere die Challenges, in dem Fall die Hausaufgaben, sollen hiermit erstellt und 
		 eingepflegt werden k\"onnen. Beim Erstellen von Hausaufgaben kann man einen Titel ausw\"ahlen und andere Eigenschaften
		 festsetzten. Zum Beispiel kann ein Zeit-Slot gesetzt werden in der die Hausaufgabe bearbeitet werden muss. 
	\item Auch die Studenten k\"onnen im Anschluss der Bearbeitung dieses Tool nutzen um zu \"uberpr\"ufen
		 ob sie die Hausaufgaben bestanden haben oder nicht. 
	\item Das Back-End bietet ebenfalls eine Schnittstelle an, die mit dem Software-Modul des Teamprojektes kommuniziert und es mit 
		entsprechenden XML-Schemata f\"ullt.
	\item Die Studenten bekommen die M\"oglichkeit Kommentare zu den Hausaufgaben abzugeben. Somit bekommt auch der Betreuer der
		Hausaufgaben eine Rewiew zum Schwierigkeitsgrad oder \"ahnlichen Kriterien. Auch dies wird durch das Back-End verwaltet.
\end{enumerate}



