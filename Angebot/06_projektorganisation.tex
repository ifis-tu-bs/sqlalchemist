%!TEX root = ../Angebot.tex

\chapter{Projektorganisation}

\section{Schnittstelle zum Auftraggeber}

Der Kunde steht den Entwicklern per Telefon-, E-Mail- und pers\"onlicher Kommunikation zur Verf\"ugung. Regelm\"a{\ss}ige
Meetings wurden direkt zu Anfang des Projektes auf donnerstags um 15 Uhr festgelegt.


\section{Schnittstelle zu anderen Projekten}

Die Mitglieder des externen Teamprojektes, die ihre Software zur Verf\"ugung stellen, und der Verfasser 
der Bachelorarbeit, dessen Software ebenfalls zur Verg\"ugung gestellt wird, k\"onnen sowohl \"uber die 
Projektbetreuer, als auch per E-Mail erreicht werden. 
Der Datenaustausch geschieht hierbei \"uber GitHub.

\section{Interne Kommunikation}

Das SEP-Team kommuniziert \"uber E-Mail bzw. Instant-Messenger und das eingerichtete Repository im Redmine.
Au{\ss}erdem finden sowohl mit den Betreuern und dem Kunden, als auch intern untereinander regelm\"a{\ss}ige Treffen statt.
Die Treffen werden mit dem Online-Termin-Planer Doodle organisiert.
Zur Protokollierung dieser Meetings wird ein webbasierter Editor von Google verwendet.


%\section{Kontaktdaten}

	
	