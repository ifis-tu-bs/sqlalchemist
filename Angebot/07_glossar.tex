%!TEX root = ../Angebot.tex

\chapter{Glossar}

\begin{itemize}
	\item Back-End: Das Back-End bezeichnet eine Schicht eines Programms. Das Back-End verarbeitet zur Verf\"ugung gestellte Daten. Es entspricht hierbei der Serverkomponente des Programms.
	\item Front-End: Das Front-End ist das Gegenst\"uck zu dem Back-End. Es beschreibt Ausgabe des Programms und ist f\"ur eine entsprechende Visualisierung zust\"andig.
	\item Javadoc: Javadoc ist ein Software-Dokumentationswerkzeug, das aus Java-Quelltexten automatisch HTML- Dokumentationsdateien erstellt. 
	\item Redmine: Das Redmine ist eine webbasierte Organisations- und Projektverwaltungssoftware.
	%!\item Sprite: Ein Sprite ist ein zweidimensionales Grafikobjet. 
	\item Repository: Ein Repository ist ein verwaltetes Verzeichnis zur Speicherung von Daten
	\item Framework: Ein Framework ist ein vorgefertigtes Programmierger\"ust.
	\item Spiele-Engine: Eine Spiel-Engine ist ein spezielles Spiele-Framework, das den Spielverlauf steuert und f\"ur die Visualisierung des Spiels sorgt.  
\end{itemize}



